% !TEX root = main.tex
\section{Summary}
\label{sec:Summary}


The $\Bs$ oscillation frequency $\dms$ is measured from the time-dependent fit to $\Bs\to\Ds\pion\pion\pion$ data to be
\begin{equation*}
\dms = 17.7651 \pm 0.0084 \pm 0.0058,  
\end{equation*}
where the errors are statistical and systematic, respectively.
Table \ref{tab:ResultSummary} summarizes the values for the ratio of the $b\to u$ and $b\to c$ transition amplitudes ($r$)
%, their coherence ($\kappa$)
and their strong ($\delta$) and weak phase ($\gamma -2 \beta_s$) difference 
%CKM angle $\gamma$, the ratio of the static total decay amplitudes $r$ 
%and the strong phase difference $\delta$ 
obtained from the phase-space integrated, as well as the time-dependent amplitude fit to $\Bs\to\Ds\kaon\pion\pion$ data.
%The values for the CP-violating parameters, measured in the phase-space integrated, time-dependent fit to $\Bs\to\Ds\kaon\pion\pion$ data, summarized in Table~\ref{tab:sigFitResults}, 
%give rise to a $xx.xx \sigma$ evidence of CP violation in the interference between mixing and decay in this channel.
%Furthermore, the masses and widths of the $K_{1}(1400)$ and the $K^{*}(1410)$ resonances are measured in the time-dependent amplitude fit:
%\begin{align*}
%m_{K_{1}(1400)} [\mevcc]       & = 1398 \pm 9 \pm 4 \pm 6,   \\
%\Gamma_{K_{1}(1400)} [\mevcc]  & =  204 \pm 14 \pm 9 \pm 9 ,  \\
% m_{K^{*}(1410)} [\mevcc]      & =   1432 \pm 12 \pm 15 \pm 8, \\
% m_{K^{*}(1410)} [\mevcc]      & =  344 \pm 25 \pm 35 \pm 18,
%\end{align*}
%where the uncertainties are statistical, systematic and due to the choice of amplitude models, respectively.
%\\
%
%\noindent 
%Table \ref{tab:compareCP} compares the difference between the physical parameters obtained from the phase-space integrated and full
%time-dependent amplitude fit. 
%In the latter case, the \CP coefficients are derived from the fit result using Eq.~\ref{eq:CPcoeff}. 
Their comparison is not so straightforward as the results are obtained from the same data set. 
We define a measure of their compatibility as follows:
For the statistical error we take the difference of the statistical uncertainties of both fits:
$\Delta \sigma_{stat} = \sigma^{MI}_{stat} - \sigma^{Full}_{stat}$ ,
where $\sigma^{MI}_{stat}$ is the statistical uncertainty of the phase-space integrated (model-independent) fit 
and $\sigma^{Full}_{stat}$ is the statistical uncertainty of the full time-dependent amplitude fit.
In the limit of equal statistical uncertainties, $\Delta \sigma_{stat} = 0$,  and the results should agree perfectly.
For the systematic error we assume that the model-dependent error (including resonance line shapes, form factors and alternative amplitude models) of the full time-dependent amplitude fit, 
$\sigma^{Full}_{model}$, is uncorrelated 
to the phase-space integrated results.
The remaining systematics (including time-acceptance, resolution, tagging, etc.) are assumed to be 100\% correlated such that their effect should cancel.
Hence, the results should agree within a spread given by:
$\Delta \sigma = \sqrt{ \Delta \sigma_{stat}^2 + (\sigma^{Full}_{model}){}^2}$.
The physical parameters agree within a range of 0.4 to 1.4 $\Delta \sigma$. 

%There is an additional caveat due to the fact that the phase-space integrated fit does not enforce the conditions 
%$1 = C^2 + D^2 + S^2 = C^2 + \bar D^2 + \bar S^2$.
%This constrain is only implemented afterwards when converting the \CP parameters to $r, \kappa, \delta, \gamma$ with the GammaCombo tool.
%The full time-dependent amplitude fit, on the other hand, has this condition already built in. 
%This means that even for the case of $\Delta \sigma = 0$, one would not expect to observe the very same results for the two methods.
%The final (unblinded) result in terms of $r, \kappa, \delta, \gamma$ will be more straightforward to compare.

\begin{table}[h]
\centering
\caption{Parameters determined from the two fits performed to the $\Bs\to\Ds\kaon\pion\pion$ data sample. 
 Statistical and systematic uncertainties are combined.}
\begin{tabular}{l r r r }
\hline
\hline
Parameter & Phase-space integrated fit & Time-dependent amplitude fit &\multicolumn{1}{c}{ Difference [$\Delta\sigma$]} \\
\hline
$r$ & 0.41 $\pm$ 0.10  & 0.50 $\pm$ 0.05 & 1.4 \\
$\kappa$ & 0.65 $\pm$ 0.21  & 0.52 $\pm$ 0.11 & 0.8\\
$\delta$ [$\degrees$] & 40 $\pm$ 18 & 46 $\pm$ 18 & 0.6\\
$\gamma - 2 \beta_s$ [$\degrees$] & 65 $\pm$ 21  & 61 $\pm$ 17 & 0.4\\
\hline
\hline
\end{tabular}
\label{tab:ResultSummary}
\end{table}

%\begin{table}[h]
%\centering
%\caption{
%Difference between the results of the \CP parameters of the phase-space integrated and full time-dependent fit.
%}
%%\resizebox{\linewidth}{!}{
%	\renewcommand{\arraystretch}{1.5}
%\begin{tabular}{c r r} 
%\hline
%\hline
%Fit Parameter & \multicolumn{1}{c}{Difference}  &\multicolumn{1}{c}{ Difference [$\Delta\sigma$]}  \\ 
%\hline
%$C$ & 0.08 & 0.7\\ 
%$D$ & 0.39 & 1.2\\ 
%$\bar{D}$ & 0.33 & 1.1\\ 
%$S$ & 0.04 & 0.3\\ 
%$\bar{S}$ & 0.14  & 0.8\\ 
%\hline
%\hline
%\end{tabular}
%\label{tab:compareCP}
%\end{table}

