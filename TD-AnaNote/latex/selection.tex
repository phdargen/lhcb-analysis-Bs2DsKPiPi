% !TEX root = main.tex

\section{Data samples and event selection}
\label{sec:Selection}

%Throughout the note, we abbreviate $\Bs\to\Ds X_{s}(\to\kaon\pion\pion)$ and $\Bs\to\Ds X_{d}(\to\pion\pion\pion)$. 
%, identifying $X_{s}\to\kaon\pion\pion$ and $X_{d}\to\pion\pion\pion$ as the various resonances through which the decays proceed. 

\subsection{Stripping and Trigger selection}

The dataset used for this analysis corresponds to $1 \invfb$ of proton-proton collision data
collected in 2011 with a centre of mass energy $\sqrt{s} = 7\tev$,  $2 \invfb$  collected in
2012 with $\sqrt{s} = 7\tev$ and $2 \invfb$  collected in
2015/2016 with $\sqrt{s} = 13\tev$.
Candidate $\Bs\to\Ds\kaon\pion\pion$ ($\Bs\to\Ds\pion\pion\pion$) decays are reconstructed using the
\textsf{B02DKPiPiD2HHHPIDBeauty2CharmLine} (\textsf{B02DPiPiPiD2HHHPIDBeauty2CharmLine})
line of the \textsf{BHadronCompleteEvent} stream of  
\textsf{Stripping21r1} (2011), \textsf{Stripping21} (2012),
\textsf{Stripping24r1} (2015)  and \textsf{Stripping28r1p1} (2016).
%and \textsf{Stripping29r2} (2017)
Both stripping lines employ the same selection cuts, listed in Appendix \ref{sec:StripAndTrigger}, except for the PID requirement on the bachelor kaon/pion.
We reconstruct the $\Bs\to\Ds h\pion\pion$ decay through two different final states of the $\Ds$ meson, $\Ds\to\kaon\kaon\pion$ and $\Ds\to\pion\pion\pion$.
Of those, $\Ds\to\kaon\kaon\pion$ is the most prominent final state,
while $\mathcal{BR}(\Ds\to\pion\pion\pion) \approx 0.2\cdot\mathcal{BR}(\Ds\to\kaon\kaon\pion)$ holds for the other one. 

Events that pass the stripping selection are further required to fulfill the following trigger requirements:
at the hardware stage, the $\Bs$ candidates are required to be TOS on the \textsf{L0Hadron} trigger or TIS on \textsf{L0Global};
at Hlt1, $\Bs$ candidates are required to be TOS on the \textsf{Hlt1TrackAllL0} (\textsf{Hlt1TrackMVA} or \textsf{Hlt1TwoTrackMVA}) trigger line for Run-I (Run-II) data;
at Hlt2, candidates have to be TOS on either one of the 2, 3 or 4-body topological trigger lines or the inclusive $\phi$ trigger. 
More details on the used HLT lines are given in Appendix \ref{sec:StripAndTrigger}.

Due to a residual kinematic dependence on whether the event is triggered by \textsf{L0Hadron} or not and on the data taking condition,
the analysis is performed in four disjoint categories: 
$[\text{Run-I,}\textsf{L0-TOS}]$, $[\text{Run-I,}\textsf{L0-TIS}]$, $ [\text{Run-II,}\textsf{L0-TOS}]$ and $ [\text{Run-II,}\textsf{L0-TIS}]$,
where for simplicity we denote $\textsf{L0-TOS}$ as $\textsf{L0Hadron-TOS}$ and $\textsf{L0-TIS} $ as $ (\textsf{L0Global-TIS} \text{ and not } \textsf{L0Hadron-TOS}$).
 


\subsection{Offline selection}

A two-fold approach is used to isolate the $\Bs\to\Ds\kaon\pion\pion$ candidates from data passing the stripping line. 
First, further one-dimensional cuts are applied to reduce the level of combinatorial background and to veto some specific physical background. 
This stage is specific to the respective final state in which the $\Ds$ meson is reconstructed, since different physical backgrounds, depending on the respective final state, have to be taken into account.   
After that, a multivariate classifier is trained which combines the information of several input variables, including their correlation, into one powerful discriminator
between signal and combinatorial background. For this stage, all possible $\Ds$ final states are treated equally. 

In order to clean up the sample and to align the Run-I to the slightly tighter Run-2 stripping selection, we apply the following loose cuts to the b hadron:
\begin{itemize}
\item DIRA $>$ 0.99994
\item min IP $\chi^{2}$ $<$ 20 to any PV,
\item FD $\chi^{2}$ $>$ 100 to any PV,
\item Vertex $\chi^{2}$/nDoF $<$ 8,
%\item ($Z_{\Ds}-Z_{\Bs}$) $>$ 0 , where $Z_{M}$ is the z-component of the position $\vec{x}$ of the decay vertex for the $\Bs$/$\Ds$ meson.
\end{itemize}    

The most prominent final state used in this analysis is $\Bs\to\Ds(\to\kaon\kaon\pion)\kaon\pion\pion$, 
where the $\Ds$ decay can either proceed via the narrow $\phiz$ resonance, the broader $\Kstarz$ resonance, or non resonant.
Depending on the decay process being resonant or not, we apply additional PID requirements on this final state:

\begin{itemize}
\item resonant case: 
\begin{itemize}
\item $\Dsp\to\phiz\pip$, with $|M(\Kp\Km) - m_{\phiz}|$ $<$ 20 $\mevcc$ : no additional requirements, since $\phiz$ is narrow and almost pure $\Kp\Km$. 
\item $\Dsp\to\Kstarzb\Kp$, with  $|M(\Km\pip) -m_{\Kstarz}|$ $<$ 75 $\mevcc$ :  $\dllkpi$ $>$ 0 for kaons, since this resonance is more than ten times broader than $\phiz$. 
\end{itemize}

\item non resonant case: $\dllkpi$ $>$ 5 for kaons, since the non resonant category has significant charmless contributions.

\end{itemize}


For the $\Ds\to\pion\pion\pion$ final state, we apply global PID requirements:

\begin{itemize}

\item $\dllkpi$ $<$ 10 for all pions.

\item $\dllppi$ $<$ 10 for all pions.

\end{itemize}
 


\subsubsection{Physics background vetoes}

Additionally, we veto various physical backgrounds, which have either the same final state as our signal decay, or can contribute via a single misidentification of $\kaon\to\pion$ or $\kaon\to\proton$. 
In the following, the vetoes are ordered by the reconstructed $\Ds$ final state they apply to: 

\begin{enumerate}

\item All:

\begin{enumerate}

\item $\Bs\to\Dsp\Dsm$ : $|M(\kaon\pion\pion) - m_{\Ds}| >$ 20 $\mevcc$.

\item $\Bs\to\Ds^{-}\Kp\Km\pip$ : possible with single missID of $\Km\rightarrow\pim$, rejected by requiring $\pim$ to fulfill $\dllkpi$ $<$ 5. 

\end{enumerate}

\item $\Ds\to\kaon\kaon\pion$

\begin{enumerate}

\item $\Bz\to\Dp(\to\Kp\pim\pip)\kaon\pion\pion$ : possible with single missID of $\pip\rightarrow\Kp$, vetoed by changing particle hypothesis and recompute $|M(\Kp\pim\pip) - m_{Dp}|$ $>$ 30 $\mevcc$, 
or the $\Kp$ has to fulfill $\dllkpi$ $>$ 10.

\item $\Lb\to\Lc(\to\proton\Km\pip)\kaon\pion\pion$ : possible with single missID of $\proton\rightarrow\Kp$, vetoed by changing particle hypothesis and recompute $M(\proton\Km\pip) - m_{\Lc}$ $>$ 30 $\mevcc$, 
or the $\Kp$ has to fulfill ($\dllkpi$ - $\dllppi$) $>$ 5.

\item $\Dz\to\kaon\kaon$ : $\Dz$ combined with a random $\pion$ can fake a $\Ds\to\kaon\kaon\pion$ decay and be a background to our signal, vetoed by requiring $M(\kaon\kaon) < 1840 \mevcc$. 

\end{enumerate}


\item $\Ds\to\pion\pion\pion$

\begin{enumerate}

\item $\Dz\to\pion\pion$ : combined with a random $\pion$ can fake a $\Ds\to\pion\pion\pion$ decay and be a background to our signal, vetoed by requiring both possible combinations to have $M(\pion\pion) < 1700 \mevcc$.

\end{enumerate}

\end{enumerate}

\subsubsection{Training of multivariate classifier}

We use TMVA \cite{Hocker:2007ht} to train a multivariate discriminator, which is used to further improve the signal to background ratio. 
The following variables are used for the training:

\begin{itemize} 

\item max(ghostProb) over all tracks

\item cone($\pt$) asymmetry of every track, 
which is defined to be the difference between the $\pt$ of the $\pi$/$\kaon$ and the sum of all other $\pt$ in a cone of radius $r = \sqrt{(\Delta\Phi)^{2} + (\Delta\eta)^{2}}$ $<$ 1 rad around the signal $\pi$/$\kaon$ track.

\item min(IP$\chi^{2}$) over the $X_{s}$ daughters

\item max(DOCA) over all pairs of $X_{s}$ daughters

\item min(IP$\chi^{2}$) over the $\Ds$ daughters

\item $\Ds$ and $\Bs$ DIRA

\item $\Ds$ FD significance

\item max($\cos(\Ds h_{i})$), where $\cos(\Ds h_{i})$ is the cosine of the angle between the $\Ds$ and another track i in the plane transverse to the beam

\item $\Bs$ IP$\chi^{2}$, FD$\chi^{2}$ and Vertex $\chi^{2}$

\end{itemize}

Various classifiers were investigated in order to select the best performing discriminator. Consequently, a boosted decision tree with gradient boost (BDTG) is chosen as nominal classifier. 
We use truth-matched MC as signal input. 
Simulated signal candidates are required to pass the same trigger, stripping and preselection requirements, that were used to select the data samples.  
For the background we use events from the high mass sideband ($m_{\Bs candidate}$ $>$ 5600 $\mevcc$) of our data samples. 
%As shown in Fig. \ref{fig:massforBDT}, this mass region is sufficiently far away from signal structures and is expected to be dominantly composed of combinatorial background.
%For completeness, the mass distribution of preselected $\Ds\to\hadron\hadron\hadron$ candidates (where $\hadron = \pion$ or $\hadron = \kaon$) is also shown in Fig. \ref{fig:massforBDT}.    \newline
%\begin{figure}[h]
%%\vspace*{-0.4cm}
%\includegraphics[height=7.0cm,width=0.49\textwidth]{figs/Ds_MM_afterpresel.pdf}
%\includegraphics[height=7.0cm,width=0.49\textwidth]{figs/Bs_MM_afterpresel.pdf}
%%\vspace*{-0.2cm}
%\caption{Invariant mass distribution of preselected (left) $\Ds\to\hadron\hadron\hadron$ and (right) $\Bs\to\Ds\kaon\pion\pion$ candidates. 
%For the  $\Bs\to\Ds\kaon\pion\pion$ candidates, the region right from the red colored line with $m_{\Bs candidate}$ $>$ 5600 $\mevcc$ is used as background input for the boosted decision tree.}
%\label{fig:massforBDT}
%\end{figure}

The distributions of the input variables for signal and background and 
the BDTG output distribution are shown in the appendix.

\begin{figure}[h]
\centering
\includegraphics[height=!,width=0.45\textwidth]{figs/MassFit/norm_preselected_pull.pdf}
\includegraphics[height=!,width=0.45\textwidth]{figs/TMVA/BDTG_Data_run1_t0_even/rejBvsS.pdf}
\caption{}
\label{fig:}
\end{figure}


\subsubsection{Final selection}


%\begin{figure}[h]
%%\vspace*{-0.4cm}
%\includegraphics[height=6.cm,width=0.95\textwidth]{figs/BDT_Input_1.pdf}
%\includegraphics[height=6.cm,width=0.95\textwidth]{figs/BDT_Input_2.pdf}
%\includegraphics[height=6.cm,width=0.95\textwidth]{figs/BDT_Input_3.pdf}
%%\vspace*{-0.2cm}
%\caption{Distributions of the input variables used in the BDTG training. The background is shown as red hatched, while the signal is depicted solid blue.}
%\label{fig:BDT_Input_1}
%\end{figure}

%The relative importance of the input variables for the BDTG training is summarized in Table \ref{table:InputVars}.
%
%\begin{table}[h]
%\centering
% \begin{tabular}{l c}
%Variable & relative importance [$\%$]\\
%  \hline
%pi\_minus\_ptasy\_1.00 & 7.32\\
%log\_Ds\_FDCHI2\_ORIVX & 7.23\\
%K\_plus\_ptasy\_1.00 & 7.17\\
%log\_Ds\_DIRA & 6.96\\
%Bs\_ENDVERTEX\_CHI2 & 6.82\\
%max\_ghostProb & 6.76\\
%pi\_plus\_ptasy\_1.00 & 6.57\\
%log\_DsDaughters\_min\_IPCHI2 & 6.21\\
%log\_Bs\_DIRA & 6.15\\
%K\_plus\_fromDs\_ptasy\_1.00 & 6.10\\
%log\_XsDaughters\_min\_IPCHI2 & 5.87\\
%K\_minus\_fromDs\_ptasy\_1.00 & 5.62\\
%cos(Ds h) & 5.58\\
%log\_Bs\_IPCHI2\_OWNPV & 5.08\\
%log\_Bs\_FDCHI2\_OWNPV & 4.04\\
%Xs\_max\_DOCA & 3.98\\
%pi\_minus\_fromDs\_ptasy\_1.00 & 2.59\\
%\end{tabular}
%\caption{Summary of the relative importance of each variable in the training of the BDTG.}
%\label{table:InputVars}
%\end{table}

 
%\begin{figure}[h]
%%\vspace*{-0.4cm}
%\includegraphics[height=7.4cm,width=0.7\textwidth]{figs/BDT_Response_new.pdf}
%%\vspace*{-0.2cm}
%\caption{BDTG output classifier distribution for (blue) signal and (red) background. The response of an independent test sample (dots) is overlaid.}
%\label{fig:BDT_Response}
%\end{figure}

       
%We determine the optimal cut value by maximizing the figure of merit $S/\sqrt{S+B}$ where S is the signal yield and B the background yield in the signal region, defined to be within $\pm$50 $\mevcc$ of the nominal $\Bs$ mass. 
%To avoid a bias in the determination of the branching fraction, we determine S and B using our normalization channel. 
%All trigger, stripping and additional selection criteria described in this and the previous chapter are applied to the $\Bs\to\Ds\pion\pion\pion$ data samples. 
%After that, we perform a simplified version of the fit to the invariant mass distribution of $\Bs\to\Ds\pion\pion\pion$ candidates described in Sec.~\ref{sec: fitting}.
%Here, a Gaussian function to model the signal and an exponential function to model combinatorial background is used.
%From this fit we estimate the number of signal events in our normalization channel. 
%Multiplying that number with the PDG branching fraction of $\frac{\mathcal{B}(\Bs\to\Ds\kaon\pion\pion)}{\mathcal{B}(\Bs\to\Ds\pion\pion\pion)}$ and the ratio of efficiencies discussed in Sec. \ref{sec: efficiency} allows us to estimate the expected number of $\Bs\to\Ds\kaon\pion\pion$ signal decays. The number of background events can then be computed as
%
%\begin{equation}
% N_{bkg}=N_{all}-N_{sig}|_{m_{\Bs\pm50\mevcc}}.   
%\end{equation}
%
%The efficiency curves as a function of the cut value are shown in Fig. \ref{fig:BDT_Efficiency}. 
%The optimal cut value is found to be BDTG $>$ 0.7012. At this working point the signal efficiency is estimated to be 72.47 $\%$, while the background rejection in the signal region is 97.38 $\%$. 


%\begin{figure}[h]
%%\vspace*{-0.4cm}
%\includegraphics[height=7.4cm,width=0.7\textwidth]{figs/BDT_CutEfficiency.pdf}
%%\vspace*{-0.2cm}
%\caption{Efficiency and purity curves for (blue) signal, (red) background and the (green) FoM curve, as a function of the chosen cut value.}
%\label{fig:BDT_Efficiency}
%\end{figure}


\clearpage



\begin{table}[h]
\centering
\caption{Offline selection requirements for $D_s \to 3 h$ candidates.}
\resizebox{\linewidth}{!}{
 \renewcommand{\arraystretch}{1.}
 \small
 \begin{tabular}{l l l}
\hline
\hline
 & Description & Requirement  \\
\hline
$D_s \to h h h$ &  $m(hhh)$ & = $m_{D_s} \pm 20 \mev$  \\
\\
$D_s^- \to K K \pi^-$  & $D^0$ veto  & $m(K K) < 1840$ MeV \\
\\
$D_s^- \to \phi \pi^-$ & $m(KK)$  & $= m_{\phi} \pm 20$ MeV \\
& PIDK($K^+$) & $> -10$ \\
& PIDK($K^-$) & $> -10$ \\
& PIDK($\pi^-$) & $< 20$ \\
&  $\chi^{2}_{FD}$ & $> 0$ \\
&  FD in $z$  &$ > -1$ \\
& $D^-$ veto  & $m(\Kp K^-_\pi \pim) \ne m(D^-) \pm 30$ MeV  $||$ $\text{PIDK}(K^-) > 0$\\
& $\Lambda_c$ veto  & $m(\Kp K^-_p \pim) \ne m(\Lambda_c) \pm 30$ MeV   $||$ $\text{PIDK}(K^-)-\text{PIDp}(K^-) > 0$ \\
\\
$D_s^- \to K^{*}(892) K^-$ & $m(KK)$  & $\ne m_{\phi} \pm 20$ MeV \\
& $m(K^+\pi^-)$  & $= m_{K^{*}(892)} \pm 75$ MeV \\
& PIDK($K^+$) & $> -10$ \\
& PIDK($K^-$) & $> -5$ \\
& PIDK($\pi^-$) & $< 10$ \\
&  $\chi^{2}_{FD}$ & $> 2$ \\
&  FD in $z$  &$ > 0$ \\
& $D^-$ veto  & $m(\Kp K^-_\pi \pim) \ne m(D^-) \pm 30$ MeV  $||$ $\text{PIDK}(K^-) > 5$\\
& $\Lambda_c$ veto  & $m(\Kp K^-_p \pim) \ne m(\Lambda_c) \pm 30$ MeV   $||$ $\text{PIDK}(K^-)-\text{PIDp}(K^-) > 5$ \\
\\
$D_s^- \to (K K \pi^-)_{NR}$ & $m(KK)$  & $\ne m_{\phi} \pm 20$ MeV \\
& $m(K^+\pi^-)$  & $\ne m_{K^{*}(892)} \pm 75$ MeV \\
& PIDK($K^+$) & $> 5$ \\
& PIDK($K^-$) & $> 5$ \\
& PIDK($\pi^-$) & $< 10$ \\
&  $\chi^{2}_{FD}$ & $> 5$ \\
&  FD in $z$  &$ > 0$ \\
& $D^-$ veto  & $m(\Kp K^-_\pi \pim) \ne m(D^-) \pm 30$ MeV  $||$ $\text{PIDK}(K^-) > 20$\\
& $\Lambda_c$ veto  & $m(\Kp K^-_p \pim) \ne m(\Lambda_c) \pm 30$ MeV  $||$ $\text{PIDK}(K^-)-\text{PIDp}(K^-) > 5$ \\
\\
$D_s \to \pi \pi \pi$ & PIDK($\pi$) & $< 10$  \\
& PIDp($\pi$) & $< 10$ \\
& $D^0$ veto  & $m(\pip \pim) < 1700$ MeV \\
&  $\chi^{2}_{FD}$ & $> 9$ \\
&  FD in $z$  &$ > 0$ \\
\\
$D_s^- \to K^- \pip \pim$ & PIDK($K$) & $> 10$  \\
& PIDK($\pi$) & $< 5$ \\
& PIDp($\pi$) & $< 10$ \\
& $D^0$ veto  & $m(K^- \pip) < 1750$ MeV \\
&  $\chi^{2}_{FD}$ & $> 9$ \\
&  FD in $z$  &$ > 0$ \\
\\
\hline
\hline
\end{tabular}
}
\label{table:selDs}
\end{table}


\begin{table}[h]
\centering
\caption{Offline selection requirements for $B_s\to D_s K \pi\pi (D_s \pi\pi\pi)$ candidates.}
%\resizebox{\linewidth}{!}{
 \renewcommand{\arraystretch}{1.}
 \small
 \begin{tabular}{l l l}
\hline
\hline
 & Description & Requirement  \\
\hline
$B_s \to D_s h \pi \pi$  & $m(D_s h \pi \pi)$ & $> 5200 \mev$ \\
&  $\chi^{2}_{vtx}/\text{ndof}  $&$ <  8$ \\
& DIRA &$ > 0.99994$ \\
& $\chi^{2}_{FD}$ & $> 100$ \\
& $\chi^{2}_{IP}$ & $< 20$ \\
&  $\chi^{2}_{DTF}/\text{ndof} $&$   <  15 $ \\
& $t$  & $ > 0.4 \ps$ \\
& $\delta t$  & $ < 0.15 \ps$ \\
& Phasespace region & $m(h\pi\pi) < 1.95 \gev$ \\ & & $m(h\pi) < 1.2 \gev$ \\ & & $m(\pi\pi) < 1.2 \gev$ \\
& Wrong PV veto & nPV = 1 $||$  $\text{min}(\Delta\chi^{2}_{IP}) > 20$ \\
\\
$X_s^+ \to K^+ \pi^+ \pi^-$  & PIDK(K) & $> 10$ \\
& PIDK($\pi^+$) & $< 10$ \\
& PIDK($\pi^-$) & $< 5$ \\
& Semi.-lep. veto & isMuon($K^+$) = 0 \\
\\
$X_s^+ \to \pi^+ \pi^+ \pi^-$  & PIDK($\pi^+$) & $< 5$ \\
& PIDK($\pi^-$) & $< 10$ \\
& Semi.-lep. veto & isMuon($\pip$) = 0 \\

\hline
\hline
\end{tabular}
%}
\label{table:selBs}
\end{table}


\subsection{Simulation}




