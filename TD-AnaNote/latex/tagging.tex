% !TEX root = main.tex
\section{Flavour Tagging}
\label{sec:Tagging}
To successfully perform a time- and amplitude-dependent measurement of $\gamma$, the identification of the initial state flavour of the $\Bs$ meson is crucial.
In the presented analysis, a number of flavour tagging algorithms are used that either determine the flavour of the non-signal b-hadron produced in the event (opposite site, OS), 
or they use particles produced in the fragmentation of the signal candidate $\Bs$/$\Bsb$ (same side, SS). \newline
For the same side, the algorithm searching for the charge of an aditional kaon that acompanies the fragmentation of the signal candidate is used (SS-nnetKaon). For the opposite site, four different taggers are chosen: 
The Two algorithms that use the charge of an electron or a muon from semileptonic B decays (OS- $\electron$,$\muon$), the tagger that uses the charge of a kaon from a b $\to$ c $\to$ s decay chain (OS-nnetKaon) 
and the algorithm that determines the $\Bs$/$\Bsb$ candidate flavour from the charge of a secondary vertex, reconstructed from the OS b decay product (OS-VtxCharge). 
All four taggers are then combined into a signel OS tagger. \newline
Every single tagging algorithm is prone to misidentify the signal candidate at a certain mistag rate $\omega = (wrong tags)/ (all tags)$. 
This might be caused by particle misidentification, flavour oscillation of the neutral opposite site B-meson or by tracks that are wrongly picked up from the underlying event. 
For every signal $\Bs$/$\Bsb$ candidate, each tagging algorithm predicts a mistag probability $\eta$, which is calculated using a combination of inputs such as the kinematics of the tagging particles. 
The inputs are then combined to a predicted mistag using neural networks. These are trained on simulated samples of $\Bs\to\Dsm\pip$ (SS algorithm) and $\Bu\to\jpsi\Kp$ (OS algorithms) decays.
For the presented analysis, the measurable CP-violating coefficients are damped by the tagging dilution $D$, that depends on the mistag rate:

\begin{equation}
\label{eq: taggingDilution}
D = 1 - 2\omega.
\end{equation}

This means that the statistical precision, with which these coeffcients can be measured, scales as the inverse square root of the effective tagging efficiency,

\begin{equation}
\label{eq: taggingEfficiency}
\epsilon_{eff} = \epsilon_{tag}(1 - 2\omega)^{2},
\end{equation}

where $epsilon_{tag}$ is the fraction of events that have a tagging decision. 
The flavour tagging algorithms are optimised for highest $\epsilon_{eff}$ on data, using the $\Bs\to\Dsm\pip$ and $\Bu\to\jpsi\Kp$ samples. \newline
Utilizing flavour-specific final states, the predicted mistag $\eta$ of each tagger has to be calibrated to match the observed mistag $\omega$ on the data sample. 
For the calibration, a linear model of the form

\begin{equation}
\label{eq: mistagCalibration}
\omega(\eta) = p_{o} + p_{1} \cdot (\eta - < \eta >), 
\end{equation}  

where the values of $p_{0}$ and $p_{1}$ are determined using the $\Bs\to\Ds\pion\pion\pion$ normalization mode and $<\eta>$ is the average estimated mistag probability $<\eta> = \Sigma_{i=1}^{N_{cand}}(\eta_{i}) / N_{cand}$.
Following this model, a perfectly calibrated tagger would lead to $\omega(\eta) = \eta$ and one would expect $p_{1} = 1$ and $p_{0} = <\eta>$.
Due to the different interaction cross-sections of oppositely charged particles, the tagging calibration parameters depend on the initial state flavour of the $\Bs$. 
Therefore, the flavour asymmetry parameters $\Delta p_{0}$, $\Delta p_{1}$ and $\Delta\epsilon_{tag}$ are introduced. 
For this analysis, the calibrated mistag is treated as per-event variable, giving a larger weight to events that are less likely to have an incorrect tag. 
This adds one additional observable to the time- and amplitude-dependent fit. \newline
The tagging calibration is determined using a time-dependent fit to the full $\Bs\to\Ds\pion\pion\pion$ sample, where the mixing frequency $\dms$ is fixed to the nominal PDG value \cite{PDG2014}.
The calibration procedure for the OS tagging algorithms (Sec.\ref{subsec: OScalibration}) and 
the SS kaon tagger (Sec.\ref{subsec: SScalibration}) is applied on the full Run I and 2015 and 2016 Run II $\Bs\to\Ds\pion\pion\pion$ data sample, which is selected following the steps described in Sec. \ref{sec:Selection}.
The similar selection ensures as close as possible agreement between the $\Bs\to\Ds\pion\pion\pion$ and $\Bs\to\Ds\kaon\pion\pion$ samples in terms of the decay kinematics, which are crucial for the flavour tagging.
Section \ref{subsec: TaggingComparison} shows the compatibility of both samples. After applying the calibration, the response of the OS and SS taggers are combined, which is shown in Sec. \ref{subsec: TaggingCombination}.  




\subsection{OS tagging calibration}
\label{subsec: OScalibration}
The responses of the OS electron, muon, neural net kaon and the secondary vertex charge taggers are combined for the mistag calibration. 
Figure \ref{fig:OSdistribution} shows the distribution of the predicted OS mistag for signal candidates from $\Bs\to\Ds\pion\pion\pion$. 
The extracted calibration parameters and tagging asymmetries are summarized in Table \ref{table: OScalibration} and the measured tagging power for the OS combination is $\epsilon_{eff,OS} = 4.81 \%$.

\begin{figure}[h]
\centering
\includegraphics[height=7.4cm,width=0.7\textwidth]{figs/Tagging/OS_combination_etaDis.pdf}
\caption{Distribution of the predicted OS combination mistag probablity for $\Bs\to\Ds\pion\pion\pion$ signal candidates.}
\label{fig:OSdistribution}
\end{figure}


\begin{table}[h]
\centering
\scriptsize
 \begin{tabular}{l l l l | l l | l}
\hline
$p_{0}$ & $p_{1}$ & $<\eta>$ & $\epsilon_{tag}$ & $\Delta p_{o}$ & $\Delta p_{1}$ & $\epsilon_{eff}$ [$\%$] \\
\hline
0.025 $\pm$0.005  & 0.944 $\pm$ 0.048 & 0.347 & 0.517 $\pm$ 0.002 & 0.028 $\pm$ 0.005 & 0.037 $\pm$ 0.045 & 4.81 $\pm$ 0.04 (stat) $\pm$ 0.37 (cal) \\
\hline
\end{tabular}
\caption{Calibration parameters and tagging asymmetries of the OS tagger extracted from $\Bs\to\Ds\pion\pion\pion$ decays.}
\label{table: OScalibration}
\normalsize
\end{table}


\subsection{SS tagging calibration}
\label{subsec: SScalibration}
The SS neural net kaon tagger can be calibrated using the flavour-specific $\Bs\to\Ds\pion\pion\pion$ decay. It's development, performance and calibration is described in detail in \cite{Aaij:2016psi}. 
Figure \ref{fig:SSdistribution} shows the distribution of the predicted mistag of the neural net kaon tagger. 
The extracted calibration parameters and tagging asymmetries are summarized in Table \ref{table: SScalibration} and the measured tagging power for this algorithm is $\epsilon_{eff,SS} = 3.22  \%$.


\begin{figure}[h]
\centering
\includegraphics[height=7.4cm,width=0.7\textwidth]{figs/Tagging/SS_nnetKaon_etaDis.pdf}
\caption{Distribution of the predicted SS neural net kaon tagger mistag probablity for $\Bs\to\Ds\pion\pion\pion$ signal candidates.}
\label{fig:SSdistribution}
\end{figure}


\begin{table}[h]
\centering
\scriptsize
 \begin{tabular}{l l l l | l l | l}
\hline
$p_{0}$ & $p_{1}$ & $<\eta>$ & $\epsilon_{tag}$ & $\Delta p_{o}$ & $\Delta p_{1}$ & $\epsilon_{eff}$ [$\%$] \\
\hline
0.008 $\pm$ 0.004  & 1.086 $\pm$ 0.059 & 0.381 & 0.571 $\pm$ 0.002 & -0.017 $\pm$ 0.004  & 0.135 $\pm$ 0.058 & 3.22 $\pm$ 0.03 (stat) $\pm$ 0.26 (cal) \\
\hline
\end{tabular}
\caption{Calibration parameters and tagging asymmetries of the SS tagger extracted from $\Bs\to\Ds\pion\pion\pion$ decays.}
\label{table: SScalibration}
\normalsize
\end{table}


\subsection{Tagging performance comparison between the signal and normalization channel}
\label{subsec: TaggingComparison}

To justify the usage of the tagging calibration, obtained using the $\Bs\to\Ds\pion\pion\pion$ sample, for our signal decay, the performance of the taggers in the two decay channels needs to be compatible. 
This is verified using both, simulated signal samples of both decays and sweighted data, 
to compare the similarity of the mistag probabilities, tagging decisions and kinematic observables that are correlated with the tagging response, on simulation and data.  \newline
The distributions of the predicted mistag probability $\eta$ for the OS combination and the SS kaon tagger are shown in Fig. \ref{fig:w_MC_comparison} (simulation) and Fig. \ref{fig:w_data_comparison} (data).
 


\begin{figure}[h]
\includegraphics[height=7.cm,width=0.49\textwidth]{figs/Tagging/w_OS_MC.pdf}
\includegraphics[height=7.cm,width=0.49\textwidth]{figs/Tagging/w_SS_MC.pdf}
\caption{Distributions of the predicted mistag $\eta$ for the OS combination (left) and the SS kaon tagger (right) in simulated $\Bs\to\Ds\kaon\pion\pion$ (black) and $\Bs\to\Ds\pion\pion\pion$ (red) signal.}
\label{fig:w_MC_comparison}
\end{figure}


\begin{figure}[h]
\includegraphics[height=7.cm,width=0.49\textwidth]{figs/Tagging/w_OS.pdf}
\includegraphics[height=7.cm,width=0.49\textwidth]{figs/Tagging/w_SS.pdf}
\caption{Distributions of the predicted mistag $\eta$ for the OS combination (left) and the SS kaon tagger (right) 
for signal candidates in the $\Bs\to\Ds\kaon\pion\pion$ (black) and $\Bs\to\Ds\pion\pion\pion$ (red) data samples. 
The signal distributions are obtained using sWeights, the procedure is described in Sec. \ref{subsec: sWegihts}.}
\label{fig:w_data_comparison}
\end{figure}


Both, data and simulated samples, show good agreement between the signal and normalization channel. 
Compatibility is also seen in Fig. \ref{fig:tagDec_data_comparison}, which shows the comparison of the tagging decision distributions of the OS and SS tagger for sweighted data. 


\begin{figure}[h]
\includegraphics[height=7.cm,width=0.49\textwidth]{figs/Tagging/qOS.pdf}
\includegraphics[height=7.cm,width=0.49\textwidth]{figs/Tagging/q_SS.pdf}
\caption{Distributions of the tagging decision from the OS combination (left) and the SS kaon tagger (right) for signal candidates in the $\Bs\to\Ds\kaon\pion\pion$ (black) and $\Bs\to\Ds\pion\pion\pion$ (red) data samples. 
The signal distributions are obtained using sWeights, the procedure is described in Sec. \ref{subsec: sWegihts}.}
\label{fig:tagDec_data_comparison}
\end{figure}

Fig. \ref{fig:kinematics_data_comparison} shows the signal data distributions of the transverse $\Bs$ momentum $\pt$, the pseudorapidity $\eta$ of the signal candidate and the number of reconstructed tracks per event.
Sufficient agreement is observed.


\begin{figure}[h]
\includegraphics[height=7.cm,width=0.49\textwidth]{figs/Tagging/Bs_Pt_comparison.pdf}
\includegraphics[height=7.cm,width=0.49\textwidth]{figs/Tagging/Bs_eta_comparison.pdf}\\
\includegraphics[height=7.cm,width=0.49\textwidth]{figs/Tagging/nTracks_comparison.pdf}
\caption{Distributions of the transverse momentum $\pt$ (top left), 
the pseudorapidity $\eta$ (top right) and the reconstructed number of tracks in the event (bottom left) for signal candidates in the $\Bs\to\Ds\kaon\pion\pion$ (black) and $\Bs\to\Ds\pion\pion\pion$ (red) data samples. 
The signal distributions are obtained using sWeights, the procedure is described in Sec. \ref{subsec: sWegihts}.}
\label{fig:kinematics_data_comparison}
\end{figure}

\begin{figure}[h]
\includegraphics[height=7.cm,width=0.49\textwidth]{figs/Tagging/Bs_Pt_norm_RunsComparison.pdf}
\includegraphics[height=7.cm,width=0.49\textwidth]{figs/Tagging/Bs_eta_norm_RunsComparison.pdf}\\
\includegraphics[height=7.cm,width=0.49\textwidth]{figs/Tagging/nTracks_norm_RunsComparison.pdf}
\caption{Distributions of the transverse momentum $\pt$ (top left), 
the pseudorapidity $\eta$ (top right) and the reconstructed number of tracks in the event (bottom left) for  $\Bs\to\Ds\pion\pion\pion$ candidates in the Run 1 (blue) and Run 2 (green) data samples. 
The signal distributions are obtained using sWeights, the procedure is described in Sec. \ref{subsec: sWegihts}.}
\label{fig:kinematics_data_comparison}
\end{figure}

To justify the portability of the flavour tagging calibration obtained from $\Bs\to\Ds\pion\pion\pion$ to the $\Bs\to\Ds\kaon\pion\pion$ channel, 
besides the good agreement of the distributions shown above, the dependence of the measured mistag $\omega$ on the predicted mistag $\eta$ has to be compatible in both channel.
This dependence is shown in Fig. \ref{fig:etavsW_mc_comparison} for simulated signal events of both channels, where good agreement is observed. 

\begin{figure}[h]
\includegraphics[height=7.cm,width=0.49\textwidth]{figs/Tagging/OS_combination_MCcomparison.pdf}
\includegraphics[height=7.cm,width=0.49\textwidth]{figs/Tagging/SS_nnetKaon_MCcomparison.pdf}
\caption{Dependence of the observed mistag $\omega$ on the predicted mistag $\eta$ for the (left) OS combination and ther (right) SS kaon tagger, 
found in the simulated $\Bs\to\Ds\kaon\pion\pion$ (black) and $\Bs\to\Ds\pion\pion\pion$ (red) signal samples.}
\label{fig:etavsW_mc_comparison}
\end{figure}


\subsection{Combination of OS and SS taggers}
\label{subsec: TaggingCombination}

In the time- and ampitude-dependent fit to $\Bs\to\Ds\kaon\pion\pion$ data, the obtained tagging responses of the OS and SS tagger will be combined after the calibration described in the previous sections is applied.
Events that aquire a mistag probability greater than 0.5 after the calibration will have their tagging decision flipped. For events where only one of the two taggers fired, the combination of the tagging decision is trivial.
In those events where both taggers made a decision, we use the standard combination of taggers \cite{LHCb-PAPER-2011-027} provided by the flavour tagging group. 
In the nominal fit, the calibrated mistags $\omega$ are combined event by event for the OS and SS tager, thus adding one variable to observable to the fit procedure. 
This ensures that the uncertainties of the OS and SS tagging calibration parameters are propagated properly to the combined tagging response for each event. \newline
The taggging performance for the combined tagger in the categories SS tagged only, OS tagged only and SS+OS tagged, are shown in Tab. \ref{tab: TaggingPerformanceTab} for the signal and normalization channel.
The distribution of the observed mistag $\omega$ as a function of the combined mistag probability $\eta$ for $\Bs\to\Ds\pion\pion\pion$ decays is shown in Fig. \ref{fig:TaggingCombinationCalibration}.


\begin{figure}[h]
\centering
\includegraphics[height=7.4cm,width=0.7\textwidth]{figs/Tagging/TaggingCombinationCalibration.pdf}
\caption{Distribution of the predicted combined mistag probablity $\eta$ versus the observed mistag $\omega$ for $\Bs\to\Ds\pion\pion\pion$ signal candidates. 
The fit with a linear polynomial, used to determine $p_{0}$ and $p_{1}$ is overlaid.}
\label{fig:TaggingCombinationCalibration}
\end{figure}


\begin{table}
\centering
\begin{tabular}{rllll}
\hline \hline
$\Bs\to\Ds\pion\pion\pion$ & \multicolumn{1}{c}{$\epsilon_{tag}$} & \multicolumn{1}{c}{$\epsilon_{eff}$} \\
\hline
SS only& $(28.586\pm0.165)\%$    & $(1.408\pm0.018(\textrm{stat})\pm0.082(\textrm{cal}))\%$\\
OS only& $(17.221\pm0.138)\%$     & $(2.027\pm0.029(\textrm{stat})\pm0.100(\textrm{cal}))\%$\\
SS+OS& $(39.981\pm0.179)\%$ & $(5.690\pm0.047(\textrm{stat})\pm0.196(\textrm{cal}))\%$\\
\hline
total & & \\
\hline \hline
$\Bs\to\Ds\kaon\pion\pion$ &  \multicolumn{1}{c}{$\epsilon_{tag}$} & \multicolumn{1}{c}{$\epsilon_{eff}$} \\
\hline
SS only& $(30.094\pm0.960)\%$    & $(1.379\pm0.082(\textrm{stat})\pm0.085(\textrm{cal}))\%$\\
OS only& $(18.923\pm0.819)\%$     & $(1.768\pm0.121(\textrm{stat})\pm0.099(\textrm{cal}))\%$\\
SS+OS& $(27.277\pm0.932)\%$ & $(3.914\pm0.194(\textrm{stat})\pm0.220(\textrm{cal}))\%$\\
\hline
total & & \\
\end{tabular}
\label{tab: TaggingPerformanceTab}
\caption{Flavour tagging performances for $\Bs\to\Ds\pion\pion\pion$ and $\Bs\to\Ds\kaon\pion\pion$ events which are only OS tagged, only SS tagged or tagged by both.}
\end{table}
