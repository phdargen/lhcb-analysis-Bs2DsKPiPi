\section{Systematic uncertainties}
\label{sec:Systematics}

This section covers all relevant systematic uncertainties on the measured observables.
In particular, the model dependent description of the invariant $\Bs$ mass spectrum, the parametrization of the time acceptance using cubic splines, 
as well as the scaling of the time resolution and tagging calibration are potential sources of systematic errors. 
The largest contribution of systematic uncertainty is expected to appear in the choice of amplitudes entering the model to describe the 5 dimensional phase space, discussed in Section \ref{sec:fullFit}.

\subsection{Models for $\Bs$ mass distribution}
\label{subsec:SystMass}

The statistical subtraction of the residual background \cite{Pivk:2004ty}, left after the full selection, relies on the correct description of the invariant $\Bs$ mass distribution.
Since the choice of signal and background models is not unique, alternative descriptions which lead to slightly different yields for the signal and background components are available. 
The difference in yields could result in shifted values for the measured observables and are therefore treated as systematic uncertainty. \newline

\subsubsection{Signal model}

The Johnson's SU function which is used as nominal signal model is replaced by a double Crystal Ball \cite{CB}. 
The crystal ball model is given by a gaussian core with an exponential tail on one side. 
Choosing a double Crystal Ball allows for asymmetric tails in a slightly different way compared to the Johnson's SU function. 
Table xXx summarizes the observed differences in signal and background yields.

\subsubsection{Background model}

For the description of the partially reconstructed background, 
a combination of the RooHILLdini and RooHORNsdini model [REF HERE] is used instead of the nominal model of three bifurcated gaussians. 
The HORNSdini model is used to describe the $\Bs\to\Ds^{*}[\to\Ds(\piz)] X_{s/d}$ decay, where the brackets around the $\piz$ indicate that it is missed in the reconstruction. 
The $\Ds^{*}\to\Ds\piz$ decay is a Vector $\to$ Scalar-Scalar ($1^{-}\to 0^{-}0^{-}$) transition. 
Using the helicity of the $\Ds$, one can show that this results in a double-peak structure in the reconstructed $\Bs$ mass. 
Therefore, the HORNSdini shape consists of a gaussian-like double-peak structure:

\begin{equation}
HORNS(m_{\Bs}) = \int^{b}_{a} dm_{\Bs}\left(m_{\Bs} - \frac{a+b}{2} \right)^{2}\mathcal{DG}(m_{\Bs}\vert\mu,\sigma,f_{G})\left(\frac{1-\zeta}{b-a}m_{\Bs} + \frac{b\zeta - a}{b - a} \right),
\label{eq:HORNS}
\end{equation}

where a and b are the kinematic endpoins of the distribution and $\zeta$ is the positive, real fraction of the two peak hights. Additionaly, the shape is convoluted with a gaussian to account for resolution effects.

The HILLdini model parametrizes the invariant mass shape of $\Bs\to\Ds^{*}[\to\Ds(\gamma)] X_{s/d}$ candidates, where the $\gamma$ is not reconstructed.
Contrary to the previously discussed process, the $Ds^{*}\to\Ds\gamma$ is a Vector $\to$ Scalar-Vector ($1^{-} \to 0^{-} 1^{-}$) transition. 
From helicity arguments, the expected shape in the mass distribution of $\Bs$ candidates follows a parabolic curve without any peaking structure.
To accommodate for this shape, the HILLdini model consists of a parabolic curve between the kinematic endpoints a \& b: 


\begin{equation}
HILL(m_{\Bs}) = \begin{cases} -(m_{\Bs} - a)(m_{\Bs} - b),& \mbox{for } a < m_{\Bs} < b \\
 0, &  otherwise. \end{cases}
\label{eq:HILLS}
 \end{equation}

This shape is convoluted with the same gaussian resolution function used for the HORNSdini model.
The resulting differences in yields is shown in Table xXx. \newline


To study systematic uncertainties originating from the description of the combinatorial background, the nominal second order polynomial is replaced by an exponential function. 
The changes in signal and background yields after refitting with this alternative shape are shown in Table xXx. \newline

\subsubsection{Systematic effect on observables}

The shift of the central values of the observables in the full fit when using sWeights obtained from a combination of alternative models, 
as well as using only one alternative model for the signal/comb.background/part.reco.background and keeping the nominal model for the other parts,
is shown in Table yYy. We conservatively choose the biggest variation as systematic uncertainty from the modelling of the invariant $\Bs$ mass spectrum.



\subsection{Decay-time acceptance}
\label{subsec:SystTime}

To investigate the systematic uncertainty related to the decay-time dependent efficiency, we vary our parametrization of the acceptance using cubic splines.
This is explicitly done by choosing slightly different knot positions, 
varying the spline coefficients at the nominal positions within their statistical uncertainties and adding/subtracting knots in the range $0.4\ps < t < 11\ps$.
Additionaly, an adaptive binning scheme is used to determine the knot positions in a way that roughly equal amounts of data is covered between two knots.
Strictly speaking, the variation of the spline coefficients within their uncertainty gives the statistical uncertainty of the decay-time acceptance parametrization.
For the presented measurement, this is done using the covariance matrix of coefficients $c_{i}$, generating toy splines with randomized coefficient values $c_{i,toy}$ from this matrix and refitting using the toy spline.  
Furthermore, the fit to the decay-time distribution of signal $\Bs\to\Ds\pion\pion\pion$ candidates, used to determine the spline parametrization, is reiterated with varying fixed/constrained values for $\DGs$.


\subsubsection{Spline parametrization}
The nominal knot positions are changed to be:

\[ k_{alt1}(t) =  {0.5 1 1.5 2 3 6 9.5}, \mbox{ } \mbox{ } k_{alt2}(t) =  {0.5  1  1.5  2  3  9 11}  \mbox{ } \mbox{ } k_{adaptive}(t) =  {0.7 1.2 1.7 2.2 6.3} \]





\subsubsection{Decay-time fit to $\Bs\to\Ds\pion\pion\pion$} 






\subsection{Decay-time resolution}

To study systematic effects originating from the scaling of the decay-time resolution $\sigma_{t}$, the decay-time distribution of fake $\Bs$ candidates using prompt $\Ds$ is described by single Gaussian function.
The resultions of the single Gaussians in the different bins of the per-event decay-time error can then be used to derive the scaling function in a straightforward way.
Since the distribution of the fake $\Bs$ decay time does not follow a perfect Gaussian distribution, two different approaches which either slightly overestimate or underestimate the decay time error are used:

\begin{itemize}

\item A double Gaussian is fit to the decay-time distributions of fake $\Bs$ candidates, but only the narrow with of the core Gaussian is considered to represent the time resolution in the respective bin. 
This method assumes that the other, broader Gaussian component does not represent the decay-time resolution of the signal $\Bs$ sample. Therefore the resolution is slightly underestimated in this case.

\item A singel Gaussian is fit to the decay-time distributions of fake $\Bs$ candidates in a wide range of $[-3\sigma_{t} : 1.5\sigma_{t}]$. 
Due to the tails of the distribution, which broaden the width of the Gaussian function, this method slightly overestimates the decay-time resolution.   

\end{itemize}

The widths of the single Gaussians from the fits performed with the two methods in bins of the per-event decay-time error is studied and a new resolution scaling function is derived for both cases: \newline

EQUATION HERE \newline

The scaling functions are shown in Fig. xXx.


    
