\section{Systematic errors}
\label{sec: systematics}

Several systematic errors contribute to the overall uncertainty on the brachning fractions. We consider the most significant ones:

\begin{itemize}

\item Particle identification

\item Signal and background models

\item Determination of the selection efficiency with MC

\item MC statistics

\item BDTG efficiency

\end{itemize} 

The particle identification (PID) efficiency is determined using PIDcalib in bins of pseudorapidity $\eta$ and transverse momentum $\pt$ of each $\Bs$ candidate. 
To estimate the systematic uncertainty, the baseline binning scheme was changed to alternative $\eta$ and $\pt$ bins. 
Tab. \ref{tab: PIDbinning} summarizes the tested binning schemes and the observed effect on $\epsilon^{pid}$. 

\begin{table}[h]
\centering
\begin{tabular}{l l l l}
 & baseline & alternative 1 & alternative 2\\
\hline
$\ptot$ bins [$\mevc$] & \scriptsize 3000-9300-15600-19000&\scriptsize 0-10000-20000-30000-40000&\scriptsize 1000-8450-15900-23350\\&\scriptsize-24400-29800-35200-40600&\scriptsize-50000-60000-70000-80000-90000&\scriptsize-30800-38250-45700-53150\\&\scriptsize-46000-51400-56800-62200&\scriptsize-100000&\scriptsize-60600-68050-75500-82950\\&\scriptsize-67600-73000-78400-83800&&\scriptsize-90400-97850-105300-112750\\&\scriptsize-89200-94600-100000&&\scriptsize-120200-127650-135100-142550\\&&&\scriptsize-150000\normalsize\\
\hline
$\eta$ bins  &\scriptsize 1.5-2.375-3.25-4.125-5 &\scriptsize 2-2.3-2.6-2.9-3.2-3.5-3.8-4.1-4.4-4.7-5 & \scriptsize 1.5-2.2-3.1-4-5\normalsize\\
\hline \hline
$\Delta\epsilon^{pid}$      & - & 0.4 $\%$ & 0.15 $\%$\\
\end{tabular}
\caption{Summary of considered binning schemes for the determination of the PID efficiency using PIDCalib.}
\label{tab: PIDbinning}
\end{table}


The maximum change in the PID efficiency due to the binning scheme is observed to be 0.4 $\%$. \newline
The systematic uncertainty arising from the mass fits is introduced by the chosen fit model and the fixed peaking background yields in the signal channel. 
Those contributions to the overall uncertainty are estimated by varying the nominal fit model and changing the expected background yield within the uncertainties given by the PIDCalib tool. 
Fixing only one of the peaking background yields (either $\Bs\to\Ds\pion\pion\pion$ or $\Bs\to\Ds^{*}\pion\pion\pion$) and floating the other one during the fit is also considered.
The variation in the yield of $\Bs\to\Ds\pion\pion\pion$ candidates is found to be neglectable ($<< 1 \%$), when a linear polynomial instead of an exponential is used to model the combinatorial background. 
Changing the signal component from a double Gaussian model to a Crystal Ball function has no significant effect on the signal yield either. 
In the signal channel, only a small change of the $N_{\Bs}$ yield is seen when a single Gaussian signal model is used instead of the nominal double gaussian. 
The most significant effect is observed when the yield of the $\Bs\to\Ds^{(*)}\pion\pion\pion$ misID background is directly determined in the fit. 
Depending on which component is floated, the signal yield increases or drops by $4 \%$. Since this is the biggest observed effect, we quote it as the uncertainty of the mass fits. \newline
The computed selection efficiency depends on how accurate the momentum spectrum of the final state particles is described by the simulation. 
To asses a potential systematic uncertainty due to the momentum modeling, we reweight the $X_{d}/X_{s}$ mass spectrum in monte carlo to agree with our observed signal data. 
Applying the weights, we observe a 0.9 $\%$ variation in the selection efficiency of $\Bs\to\Ds\kaon\pion\pion$ candidates, while no significant change can be found in the efficiency of the 
$\Bs\to\Ds\pion\pion\pion$ channel. \newline   
The uncertainty on the BDTG efficiency is determined by a fit to the $\Bs\to\Ds\pion\pion\pion$ invariant mass distribution with and without the BDTG cut. 
The maximum dissagrement is found to be 1.9 $\%$ and is assigned as the uncertainty on the BDTG efficiency. \newline
The uncertainty due to the limited MC statistic is 1.3 $\%$. \newline
All systematic uncertainties are summarized in Table \ref{tab: systTab}. The quadratic sum of all contributions is 4.7 $\%$.  

\begin{table}[h!]
\centering
\begin{tabular}{l c}
Source  & Uncertainty on $\frac{\mathcal{B}(\Bs\to\Ds\kaon\pion\pion)}{\mathcal{B}(\Bs\to\Ds\pion\pion\pion)}$ [$\%$]\\
\hline
PID & 0.4 $\%$ \\
Mass fits & 4.0 $\%$\\
MC efficiency determination & 0.9 $\%$\\
BDTG efficiency & 1.9 $\%$ \\
MC statistics & 1.3 $\%$ \\
\hline
Total & 4.7 $\%$\\
\hline
\end{tabular}
\caption{Summary of considered systematic uncertainties on the branching ratio determination.}
\label{tab: systTab}
\end{table}
