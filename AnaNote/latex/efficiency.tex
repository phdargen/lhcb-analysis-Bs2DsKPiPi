\section{Efficiency corrections}
\label{sec: efficiency}

Several relative efficiency corrections are needed to measure the branching fraction of $\Bs\to\Ds\kaon\pion\pion$ with respect to $\Bs\to\Ds\pion\pion\pion$. Precise knowledge of the efficiency related to the detector acceptance, PID requirements, used trigger lines and offline selections are crucial for both, the determination of $\gamma$ and the branching ratio measurement.

\subsection{Relative efficiency for BR measurement}
For the branching ratio measurement, the relative efficiency is given by

\begin{equation} 
\epsilon_{rel} = \epsilon^{acc}_{rel}\cdot\epsilon^{sel}_{rel}\cdot\epsilon^{pid}_{rel},
\label{eq: relEff}
\end{equation}

where $\epsilon = \frac{\epsilon_{Norm}}{\epsilon_{Sig}}$ is the ratio of the efficiency for the signal and normalization mode. To evaluate these efficiencies, we rely on simulation. The three efficiencies given in Eq. \ref{eq: relEff} are:

\begin{itemize}

\item $\epsilon^{acc}_{rel}$: This is the relative efficiency due to the geometrical acceptance of the LHCb detector. All tracks are required to have a polar angle between 10 and 400 mrad and a minimal momentum of $|p| >$ 1.6 $\gevc$ in order to be recorded for further analysis. Since the particle species of one track differs between the signal and normalizaton mode, the efficiencies caused by the geometrical acceptance are expected to be different for the two channels.

\item $\epsilon^{sel}_{rel}$: The relative selection efficiency due to trigger and offline requirements. 

\item $\epsilon^{pid}_{rel}$: The relative PID efficiency due to the identification likelihood requirements for tracks from both modes. This is evaluated using efficiencies from $\D^{*+}\to D^{0}(\Km\pip)\pip$ calibration data, 
which is weighted by the expected momentum (p) distribution taken from simulation.

\end{itemize}

 Using the definition given in Eq. \ref{eq: relEff}, the branching ratio can be expressed as

\begin{equation}
\frac{\mathcal{B}(\Bs\to\Ds\kaon\pion\pion)}{\mathcal{B}(\Bs\to\Ds\pion\pion\pion)} = \frac{\mathcal{Y}(\Bs\to\Ds\kaon\pion\pion)}{\mathcal{Y}(\Bs\to\Ds\pion\pion\pion)},
\cdot \epsilon_{rel}
\label{eq: BRwEff}
\end{equation} 

where $\mathcal{Y}(x)$ represents the yield of the respective channel. \newline
The single efficiencies, as well as the total selection efficiency, for the signal and normalization channel, is given in Table \ref{tab: effTab}.

\begin{table}[h!]
\centering
\begin{tabular}{l c c}
Efficiency ($\%$) & $\Bs\to\Ds\kaon\pion\pion$ & $\Bs\to\Ds\pion\pion\pion$\\
\hline
2011 $\epsilon^{acc}$ & 15.84 $\pm$ 0.04 & 9.85 $\pm$ 0.04\\
2012 $\epsilon^{acc}$ & 16.11 $\pm$ 0.04 & yyy\\
2011 $\epsilon^{sel}$ & 0.877 $\pm$ 0.013& 0.980\\
2012 $\epsilon^{sel}$ & 0.801 $\pm$ 0.009& yyy\\
2011 $\epsilon^{pid}$ & 74.88 $\pm$ 0.85 & 92.64 $\pm$ 0.47\\
2012 $\epsilon^{pid}$ & 74.30 $\pm$ 0.85 & -\\
\hline
2011 total $\epsilon$ & 0.110 & zz\\
2012 total $\epsilon$ & 0.101 & zz\\
\hline
\end{tabular}
\caption{Efficiencies due to the detector accpetance, selection requirements and PID cuts for the signal and normalization mode. All values are obtained using simulated events.}
\label{tab: effTab}
\end{table}





