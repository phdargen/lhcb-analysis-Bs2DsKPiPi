\section{Appendix}
\label{sec: appendix}

\subsection{Re-weghted MC observables}
\label{sec: mcvdata}
The following figures show the distributions of observables used during the multivariate selection stage, for data, monte carlo and re-weighted monte carlo.


%\begin{figure}[h]
%\includegraphics[height=6.cm,width=0.45\textwidth]{figs/.pdf}
%\includegraphics[height=6.cm,width=0.45\textwidth]{figs/.pdf}\\
%\includegraphics[height=6.cm,width=0.45\textwidth]{figs/.pdf}
%\includegraphics[height=6.cm,width=0.45\textwidth]{figs/.pdf}
%\caption{Comparison of data and simulated observables, before and after re-weighting 1.}
%\end{figure}


\subsection{Toys for normalization fit}
\label{sec: toysNorm}

To validate the fit model used to describe the $m(\Ds\pion\pion\pion)$ spectrum, we produce 1000 pseudo samples from our fit pdf and fit them with the same nominal pdf model. \newline
A pull of a certain fit parameter is defined as

\begin{equation}
p = \frac{x - x_{0}}{\Delta x},
\label{eq: pull}
\end{equation}  

where $x$ is the fitted value, $x_{0}$ the generated value and $\Delta x$ the uncertainty on $x$. 
Given the fit is correctly implemented and unbiased, one expects the distribution of the pulls for every fit parameter to be centered around 0, with a gaussian width of 1.


%\begin{figure}[h]
%\includegraphics[height=6.cm,width=0.45\textwidth]{figs/.pdf}
%\includegraphics[height=6.cm,width=0.45\textwidth]{figs/.pdf}\\
%\includegraphics[height=6.cm,width=0.45\textwidth]{figs/.pdf}
%\includegraphics[height=6.cm,width=0.45\textwidth]{figs/.pdf}
%\caption{Pull distributions for 1000 pseudo experiments, using the fit model for the normalization fit. Part 1.}
%\end{figure}
