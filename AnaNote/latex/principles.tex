\section{General principles}

The main goal is for a paper to be clear. It should be as brief as
possible, without sacrificing clarity. For all public documents,
special consideration should be given to the fact that the reader will
be less familiar with \lhcb than the author.

Here follow a list of general principles that should be adhered to:
\begin{enumerate}

\item Choices that are made concerning layout and typography
  should be consistently applied throughout the document.

\item Standard English should be used (British rather than American)
  for LHCb notes and preprints. Examples: colour, flavour, centre,
  metre, modelled and aluminium. Words ending on -ise or -isation
  (polarise, hadronisation) can be written with -ize or -ization ending.
  The punctuation normally follows the closing
  quote mark of quoted text, rather than being included before the
  closing quote.
  Footnotes come after punctuation. 
  Papers to be submitted to an American journal can be written in American
  English instead. Under no circumstance should the two be mixed.

\item Use of jargon should be avoided where possible. ``Systematics'' are ``systematic
  uncertainties'', ``L0'' is ``hardware trigger'', ``penguin'' diagrams
  are best introduced with an expression like ``electroweak loop (penguin) diagrams''.

\item Avoid using quantities that are internal jargon and/or are impossible to reproduce without the full simulation:
instead of `It is required that $\chisqvtx<3$', say 'A good quality vertex is required';
instead of `It is required that $\chisqip>16$', say `The track is inconsistent with originating from a PV';
instead of `A DLL greater than 20 is required' say `Tracks are required to be identified as kaons'.


\item Latex should be used for typesetting. Line numbering should be
  switched on for drafts that are circulated for comments.

\item The abstract should be concise, and not include citations or
  numbered equations, and should give the key results from the paper.

\item Apart from descriptions of the detector, the trigger and the
  simulation, the text should not be cut-and-pasted from other sources
  that have previously been published.

\item References should usually be made only to publicly accessible
  documents. References to LHCb conference reports and public notes
  should be avoided in journal publications, instead including the
  relevant material in the paper itself.

\item The use of tenses should be consistent. It is recommended to
  mainly stay in the present tense, for the abstract, the description
  of the analysis, \etc; the past tense is then used where necessary,
  for example when describing the data taking conditions.

\item It is recommended to use the passive rather than active voice:
  ``the mass is measured'', rather than ``we measure the mass''.
  Limited use of the active voice is acceptable, in situations where
  re-writing in the passive form would be cumbersome, such as for the
  acknowledgements.  Some leeway is permitted to accommodate different
  author's styles, but ``we'' should not appear excessively in the
  abstract or the first lines of introduction or conclusion.

\item A sentence should not start with a variable, a particle or an acronym. A title or caption should not start with an article. 

\end{enumerate}
