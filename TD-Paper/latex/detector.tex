\section{Detector and simulation}
\label{sec:Detector}
%The paragraph below can be used for the detector
%description. Modifications may be required in specific papers to fit
%within page limits, to enhance particular detector elements or to
%introduce acronyms used later in the text. For journals where strict
%word counts are applied (for example, PRL), and space is at a premium,
%it may be sufficient to write, as a minimum: ``The LHCb detector is a 
%single-arm forward spectrometer covering the pseudorapidity range 
%$2 < \eta < 5$, 
%$described in detail in Refs.~\cite{Alves:2008zz,LHCb-DP-2014-002}''. 
%A slightly longer version could specify the most relevant sub-detectors, {\it e.g} 
%``The LHCb 
%detector~\cite{Alves:2008zz,LHCb-DP-2014-002} is a
%single-arm forward spectrometer covering the pseudorapidity range $2 < \eta < 5$, designed for
%the study of particles containing \bquark\ or \cquark\ quarks. The detector elements that are particularly
%relevant to this analysis are: a silicon-strip vertex detector surrounding the $pp$ interaction
%region that allows \cquark\ and \bquark\ hadrons to be identified from their characteristically long
%flight distance; a tracking system that provides a measurement of the momentum, $p$, of charged
%particles; and two ring-imaging Cherenkov detectors that are able to discriminate between
%different species of charged hadrons.'' LHCb-PUB-2014-046

%\begin{verbatim}
%In the following paragraph, references to the individual detector 
%performance papers are marked with a * and should only be included 
%if the analysis relies on numbers or methods described in the specific 
%papers. Otherwise, a reference to the overall detector performance 
%paper~\cite{LHCb-DP-2014-002} will suffice. Note also that the text 
%defines the acronyms for primary vertex, PV, and impact parameter, IP. 
%Remove either of those in case it is not used later on.
%\end{verbatim}

The \lhcb detector~\cite{Alves:2008zz,LHCb-DP-2014-002} is a single-arm forward
spectrometer covering the \mbox{pseudorapidity} range $2<\eta <5$,
designed for the study of particles containing \bquark or \cquark
quarks. The detector includes a high-precision tracking system
consisting of a silicon-strip vertex detector surrounding the $pp$
interaction region~\cite{LHCb-DP-2014-001}, a large-area silicon-strip detector located
upstream of a dipole magnet with a bending power of about
$4{\mathrm{\,Tm}}$, and three stations of silicon-strip detectors and straw
drift tubes~\cite{LHCb-DP-2013-003,LHCb-DP-2017-001} placed downstream of the magnet.
The polarity of the dipole magnet can be reversed, which is done periodically throughout the data-taking process to control systematic asymmetries.
The tracking system provides a measurement of the momentum, \ptot, of charged particles with
a relative uncertainty that varies from 0.5\% at low momentum to 1.0\% at 200\gevc.
The minimum distance of a track to a primary vertex (PV), the impact parameter (IP), 
is measured with a resolution of $(15+29/\pt)\mum$,
where \pt is the component of the momentum transverse to the beam, in\,\gevc.
Different types of charged hadrons are distinguished using information
from two ring-imaging Cherenkov detectors~\cite{LHCb-DP-2012-003}. 
The online event selection is performed by a trigger~\cite{LHCb-DP-2012-004}, 
which consists of a hardware stage, based on information from the calorimeter and muon
systems, followed by a software stage, which applies a full event
reconstruction.

%The trigger description has to be specific for the analysis in
%question. In general, you should not attempt to describe the full
%trigger system. Below are a few variations that inspiration can be
%taken from. First from a hadronic analysis, and second from an
%analysis with muons in the final state. In case you have to look 
%up specifics of a certain trigger, a detailed description of the trigger 
%conditions for Run 1 is available in Ref.~\cite{LHCb-PUB-2014-046}. 
%{\bf Never cite this note in a PAPER or CONF-note.} 

At the hardware trigger stage, events are required to have a muon with high \pt or a
hadron, photon or electron with high transverse energy in the calorimeters. For hadrons,
the transverse energy threshold is 3.5\gev.
The software trigger requires a two-, three- or four-track
secondary vertex with a significant displacement from any primary
$pp$ interaction vertex. At least one charged particle
must have a transverse momentum $\pt > 1.6\gevc$ and be
inconsistent with originating from a PV.
A multivariate algorithm~\cite{BBDT} is used for
the identification of secondary vertices consistent with the decay
of a \bquark hadron.

%An example to describe the use of both TOS and TIS candidates:
%\begin{itemize}
%\item In the offline selection, trigger signals are associated with reconstructed particles.
%Selection requirements can therefore be made on the trigger selection itself
%and on whether the decision was due to the signal candidate, other particles produced in the $pp$ collision, or a combination of both.
%\end{itemize}

%Before describing the simulation, explain in one sentence why simulation is needed.
%The following paragraph can act as inspiration but
%with variations according to the level of detail required and if
%mentioning of \eg \photos is required.

Simulation is necessary to model the effects of the detector acceptance and to optimize the selection requirements.
In the simulation, $pp$ collisions are generated using
\pythia~\cite{Sjostrand:2006za,*Sjostrand:2007gs} with a specific \lhcb
configuration~\cite{LHCb-PROC-2010-056}.  Decays of hadrons
are described by \evtgen~\cite{Lange:2001uf}, in which final-state
radiation is generated using \photos~\cite{Golonka:2005pn}. The
interaction of the generated particles with the detector, and its response,
are implemented using the \geant
toolkit~\cite{Allison:2006ve, *Agostinelli:2002hh} as described in
Ref.~\cite{LHCb-PROC-2011-006}.


%A quantity often used in LHCb analyses is \chisqip. When mentioning it in 
%a paper, the following wording could be used: ``$\ldots$\chisqip\ with respect 
%to any primary interaction vertex greater than X, where \chisqip\ is defined as 
%the difference in the vertex-fit \chisq of a given PV reconstructed with and
%without the track under consideration/being considered.''\footnote{If this
%sentence is used to define \chisqip\ for a composite particle instead of 
%for a single track, replace ``track'' by ``particle'' or ``candidate''}

%Many analyses depend on boosted decision trees. It is inappropriate to
%use TMVA~\cite{Hocker:2007ht,*TMVA4} as sole reference as that is
%merely an implementation of the BDT algorithm.
%Rather it is suggested to write: ``In this paper we use a 
%boosted decision tree~(BDT)~\cite{Breiman,AdaBoost} implemented in the TMVA
%toolkit~\cite{Hocker:2007ht,*TMVA4} to separate signal from background''. 

%When describing the integrated luminosity of the data set, do not use
%expressions like ``1.0\invfb of data'', but \eg 
%``data sample corresponding to an integrated luminosity of 1.0\invfb'', 
%or ``a sample of data obtained from 3\invfb of integrated luminosity''. 

%For analyses where the periodical reversal of the magnetic field is crucial, 
%\eg in measurements of direct \CP violation, the following description can be
%used as an example phrase: 
%``The magnetic field deflects oppositely charged particles in opposite
%directions and this can lead to detection asymmetries. Periodically
%reversing the magnetic field polarity throughout the data-taking almost cancels
%the effect. The configuration with the magnetic field pointing upwards (downwards), 
%\MagUp (\MagDown), bends positively (negatively) charged particles
%in the horizontal plane towards the centre of the LHC ring.''
%Only use the \MagUp, \MagDown symbols if they are used extensively in tables or figures.
