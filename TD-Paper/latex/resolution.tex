\section{Decay-time resolution}
\label{sec:Resolution}

The CP-violating parameters measured in the time-dependent fit are prone to dilution due to the fast $\Bs$-$\Bsb$ oscillation frequency, 
which is of the same order as the average decay-time resolution of the LHCb detector of $\mathcal{O}(50 \fs^{-1})$~\cite{LHCb-DP-2014-002}.
Therefore, it is crucial to correctly describe the decay-time resolution in order to accurately measure time-dependent CP violation.
In particular, the parameters related to the amplitudes of the sine and cosine terms in Equation xXx are highly corelated to the chosen resolution model.
Since the time resolution depends on the particular event, especially the decay time itself,
the sensitivity on the CP parameters can be significantly improved by using an event-dependent model rather than an average resolution.
For this purpose, the signal PDF is convolved with a Gaussian resolution function that has a different width for each candidate, 
obtained from the global kinematic fit to the $\Bs$ vertex and the $\Ds$ mass.
To ensure the correct application, the per-candidate decay-time uncertainty $\sigma_{t}$ has to be calibrated to match the effective decay-time resolution observed in data, $\sigma(\sigma_{t})$.\newline
For data taken during Run I, 
a study of simulated $\Bs\to\Ds\kaon\pion\pion$ events is used to confirm the portability of the calibration relation determined in the closely related analysis of $\Bs\to\Ds\kaon$ decays~\cite{Aaij:2017lff}.
The spread of the difference between the reconstructed and true decay time, $\Delta t = t - t_{\text{true}}$, 
follows the shape of a double Gaussian distribution and is a direct measure of the effective decay-time resolution for simulated events. 
The resulting two Gaussian widths are combined to calculate the dilution $\mathcal{D}$, which describes the effective damping of the CP amplitudes due to the finite time resolution:
\begin{equation}
\mathcal{D} = f_{1}e^{-\sigma_{1}^{2}\dms^{2}/2} + (1 - f_{1}) e^{-\sigma_{2}^{2}\dms^{2}/2},
\label{eq:t-dilution}
\end{equation}
where $\sigma_{1}$ and $\sigma_{2}$ are the widths of the Gaussians, $f_{1}$ is the relative fraction of events described by the first Gaussian relative to the second and $\dms$ is the oscillation frequency of $\Bs$ mesons.
An effective single Gaussian width is calculated from the dilution as,
\begin{equation}
\sigma_{eff} = \sqrt{(-2/\dms^{2})\ln{\mathcal{D}}},
\label{eq:effres}
\end{equation}
which converts the resolution into a single-Gaussian function with an effective resolution that causes the same damping effect on the magnitude of the $B_s$ oscillation.
The calibration relation is found to be portable between the $\Bs\to\Ds\kaon$ and $\Bs\to\Ds\kaon\pion\pion$ decay channels and thus it is used for data taken in Run I.\newline
For data taken during Run II, the calibration is performed using a sample of prompt $\Ds$ mesons, 
combined with a kaon and two pions originating from the primary vertex to form 'fake' $\Bs$ candidates with a lifetime of $t = 0$ by construction. 
The spread of observed decay times of the 'fake' candidates is described by a double Gaussian function, 
where only negative decay times are used to determine the effective resolution to avoid uncertainties introduced by physical backgrounds. 
Following the same approach used for data taken during Run I, the effective resolution is calculated from the dilution $\mathcal{D}$. 
%The scaling relation is found to be $\sigma^{15,16}_{eff} = 0.88\sigma_{t} + 11.6 \fs$ and $\sigma^{17}_{eff} = 0.96\sigma_{t} + 6.5 \fs$ for data taken in 2015, 2016 and 2017, respectively.

