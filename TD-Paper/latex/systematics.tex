\section{Systematic uncertainties}
\label{sec:Systematics}

Systematic uncertainties derive from the modelling of the background in the invariant $\Bs$ mass distribution, the detection and production asymmetries $A_{det}$ and $A_{p}$, 
the limited knowledge of the decay-time acceptance and resolution, as well as from the uncertainty on the LHCb length and momentum scale, which directly translates in an uncertainty on $\dms$.
For the time-dependent amplitude fit, additional sources of systematic uncertainties arise from the description of the phase-space acceptance, the modelling of resonance shapes and the explicit choice of amplitudes used in the fit.
The systematic uncertainties on the measured observables are summarized in Table \ref{tab:sigSys} for the phase-space integrated decay-time fit
and in Table \ref{tab:sigSys2} for the full time-dependent amplitude fit to $\Bs\to\Ds\kaon\pion\pion$ data. 
The individual contributions are discussed below.\newline
Since the choice of signal and background models for the description of the invariant mass spectrum of $\Bs$ candidates is not unique, several alternative parametrizations are tested. 
For each case new signal weights are obtained and the sFit procedure is repeated. 
The sample variance of the obtained differences to the nominal fit value are assigned as systematic uncertainty due to the background subtraction.\newline 
The fit procedure is validated using a large set of pseudoexperiments, which are generated with the central values of the \CP parameters reported in Table \ref{tab:sigFitResults}. 
Subsequently, they are processed by the nominal fit procedure and the values obtained by the fits are compared to the generated ones. 
For each parameter, a distribution is formed by normalizing the differences between fitted and generated values to the uncertainties measured in the nominal fit. 
The mean and the width of the distribution is added in quadrature and assigned as systematic uncertainty due to a fit bias for the respective parameter.\newline 
The systematic uncertainty related to the decay-time acceptance, as well as $\Gamma_s$ and $\Delta\Gamma_s$ are studied with the same set of pseudoexperiments. 
They are fit with the nominal model and a model in which the acceptance parameters together with $\Gamma_s$ and $\Delta\Gamma_s$ are randomized within their uncertainties.
Distributions are calculated by dividing the difference between the obtained values of the nominal fit and the fit using randomly shifted acceptance parameters by the uncertainty in the nominal fit.
The bias in the mean of this distribution is added to its width, in quadrature, in order to arrive at the final systematic uncertainty for each parameter.\newline
This procedure is repeated, varying the production, detection asymmetries and $\dms$ within their respective uncertainties instead of the acceptance parametrization.\newline
To study systematic effects originating from the scaling of the decay-time error estimate, two alternative decay-time resolution models are tested.
Due to the high correlation between the decay-time resolution and the tagging calibration, their systematic uncertainty needs to be studied simultaneously.
First, the decay-time dependent fit to $\Bs\to\Ds\pion\pion\pion$ data is repeated using a alternative decay-time error scaling function.
In this fit, new tagging calibration parameters are obtained and subsequently used with Gaussian-constrains in the fit to $\Bs\Ds\kaon\pion\pion$ data.
The largest change in the central value of each \CP observable is assigned as the systematic uncertainty due to the decay-time resolution and flavour tagging for the respective parameter.\newline
A possible systematic effect is studied by repeating the sFit, randomly keeping only one candidate in events where multiple candidates are found. No shift in the nominal fit values is observed.\newline
The uncertainty on the LHCb length scale is estimated to be at most $0.020\%$, which translates directly in an uncertainty on $\dms$ of $0.020\%$ with other parameters being unaffected.\newline

DESCRIPTION OF AMPLITUDE SYSTEMATICS HERE

%\begin{sidewaystable}[h]
%\centering
%\caption{Systematic uncertainties on the fit parameters of the fit to $B_s \to D_s \pi \pi \pi$ data in units of statistical standard deviations.}
%\resizebox{0.75\linewidth}{!}{
%        \renewcommand{\arraystretch}{1.5}
%        \begin{tabular}{l  c  c  c  c  c  c  c  c  | c }
\hline
\hline
Fit Parameter & Fit-bias & Background & Correlations & Acceptance & Resolution & Decay-time bias & Asymmetries & Mom./z-Scale &  Total  \\ 
\hline
$p_{0}^{OS} \, \text{Run-I}$ & 0.05 & 0.09 & 0.03 & 0.01 & 1.00 & 0.02 & 0.00 &  & 1.00 \\ 
$p_{1}^{OS} \, \text{Run-I}$ & 0.01 & 0.13 & 0.06 & 0.01 & 1.04 & 0.02 & 0.01 &  & 1.05 \\ 
$\Delta p_{0}^{OS} \, \text{Run-I}$ & 0.14 & 0.03 & 0.03 & 0.00 & 0.02 & 0.01 & 0.15 &  & 0.21 \\ 
$\Delta p_{1}^{OS} \, \text{Run-I}$ & 0.07 & 0.06 & 0.16 & 0.00 & 0.03 & 0.01 & 0.15 &  & 0.24 \\ 
$\epsilon_{\text{tag}}^{OS} \, \text{Run-I}$ & 0.06 & 0.17 & 0.12 & 0.00 & 0.00 & 0.00 & 0.01 &  & 0.22 \\ 
$\Delta \epsilon_{\text{tag}}^{OS} \, \text{Run-I}$ & 0.04 & 0.01 & 0.05 & 0.00 & 0.06 & 0.02 & 0.01 &  & 0.09 \\ 
$p_{0}^{SS} \, \text{Run-I}$ & 0.03 & 0.03 & 0.15 & 0.01 & 0.56 & 0.02 & 0.00 &  & 0.58 \\ 
$p_{1}^{SS} \, \text{Run-I}$ & 0.10 & 0.03 & 0.04 & 0.01 & 0.60 & 0.01 & 0.01 &  & 0.61 \\ 
$\Delta p_{0}^{SS} \, \text{Run-I}$ & 0.04 & 0.01 & 0.07 & 0.00 & 0.00 & 0.00 & 0.10 &  & 0.13 \\ 
$\Delta p_{1}^{SS} \, \text{Run-I}$ & 0.03 & 0.04 & 0.04 & 0.00 & 0.01 & 0.00 & 0.12 &  & 0.14 \\ 
$\epsilon_{\text{tag}}^{SS} \, \text{Run-I}$ & 0.02 & 0.02 & 0.10 & 0.00 & 0.00 & 0.00 & 0.01 &  & 0.11 \\ 
$\Delta \epsilon_{\text{tag}}^{SS} \, \text{Run-I}$ & 0.04 & 0.03 & 0.02 & 0.00 & 0.04 & 0.03 & 0.01 &  & 0.07 \\ 
$p_{0}^{OS} \, \text{Run-II}$ & 0.02 & 0.20 & 0.01 & 0.02 & 0.97 & 0.04 & 0.00 &  & 0.99 \\ 
$p_{1}^{OS} \, \text{Run-II}$ & 0.02 & 0.08 & 0.10 & 0.01 & 0.60 & 0.04 & 0.00 &  & 0.61 \\ 
$\Delta p_{0}^{OS} \, \text{Run-II}$ & 0.07 & 0.08 & 0.13 & 0.00 & 0.22 & 0.01 & 0.00 &  & 0.27 \\ 
$\Delta p_{1}^{OS} \, \text{Run-II}$ & 0.02 & 0.03 & 0.16 & 0.00 & 0.08 & 0.01 & 0.00 &  & 0.18 \\ 
$\epsilon_{\text{tag}}^{OS} \, \text{Run-II}$ & 0.01 & 0.16 & 0.04 & 0.00 & 0.11 & 0.00 & 0.00 &  & 0.20 \\ 
$\Delta \epsilon_{\text{tag}}^{OS} \, \text{Run-II}$ & 0.05 & 0.05 & 0.10 & 0.00 & 0.03 & 0.03 & 0.00 &  & 0.13 \\ 
$p_{0}^{SS} \, \text{Run-II}$ & 0.10 & 0.06 & 0.05 & 0.01 & 0.61 & 0.02 & 0.00 &  & 0.62 \\ 
$p_{1}^{SS} \, \text{Run-II}$ & 0.01 & 0.07 & 0.06 & 0.02 & 0.63 & 0.04 & 0.00 &  & 0.64 \\ 
$\Delta p_{0}^{SS} \, \text{Run-II}$ & 0.07 & 0.02 & 0.01 & 0.00 & 0.04 & 0.01 & 0.00 &  & 0.09 \\ 
$\Delta p_{1}^{SS} \, \text{Run-II}$ & 0.11 & 0.05 & 0.14 & 0.00 & 0.04 & 0.01 & 0.00 &  & 0.19 \\ 
$\epsilon_{\text{tag}}^{SS} \, \text{Run-II}$ & 0.03 & 0.03 & 0.09 & 0.00 & 1.68 & 0.00 & 0.00 &  & 1.68 \\ 
$\Delta \epsilon_{\text{tag}}^{SS} \, \text{Run-II}$ & 0.01 & 0.03 & 0.01 & 0.00 & 0.21 & 0.03 & 0.00 &  & 0.21 \\ 
$A_{P} \, \text{Run-II}$ & 0.04 & 0.02 & 0.17 & 0.00 & 0.19 & 0.03 & 0.01 &  & 0.26 \\ 
$\Delta m_{s}$ & 0.01 & 0.11 & 0.02 & 0.00 & 0.13 & 0.69 & 0.02 & 0.67 & 0.98 \\ 
\hline
\hline
\end{tabular}

%}
%\label{tab:normSys}
%\end{sidewaystable}

\begin{table}[h]
\centering
\caption{Systematic uncertainties on the fit parameters of the phase-space integrated fit to $B_s \to D_s K\pi \pi$ data in units of statistical standard deviations.}
\resizebox{\linewidth}{!}{
        \renewcommand{\arraystretch}{1.5}
        \begin{tabular}{l  c  c  c  c  c  c  c  c  | c }
\hline
\hline
Fit Parameter & Fit-bias & Background & Correlations & Acceptance & Resolution & Decay-time bias & Asymmetries & Mom./z-Scale &  Total  \\ 
\hline
$p_{0}^{OS} \, \text{Run-I}$ & 0.05 & 0.09 & 0.03 & 0.01 & 1.00 & 0.02 & 0.00 &  & 1.00 \\ 
$p_{1}^{OS} \, \text{Run-I}$ & 0.01 & 0.13 & 0.06 & 0.01 & 1.04 & 0.02 & 0.01 &  & 1.05 \\ 
$\Delta p_{0}^{OS} \, \text{Run-I}$ & 0.14 & 0.03 & 0.03 & 0.00 & 0.02 & 0.01 & 0.15 &  & 0.21 \\ 
$\Delta p_{1}^{OS} \, \text{Run-I}$ & 0.07 & 0.06 & 0.16 & 0.00 & 0.03 & 0.01 & 0.15 &  & 0.24 \\ 
$\epsilon_{\text{tag}}^{OS} \, \text{Run-I}$ & 0.06 & 0.17 & 0.12 & 0.00 & 0.00 & 0.00 & 0.01 &  & 0.22 \\ 
$\Delta \epsilon_{\text{tag}}^{OS} \, \text{Run-I}$ & 0.04 & 0.01 & 0.05 & 0.00 & 0.06 & 0.02 & 0.01 &  & 0.09 \\ 
$p_{0}^{SS} \, \text{Run-I}$ & 0.03 & 0.03 & 0.15 & 0.01 & 0.56 & 0.02 & 0.00 &  & 0.58 \\ 
$p_{1}^{SS} \, \text{Run-I}$ & 0.10 & 0.03 & 0.04 & 0.01 & 0.60 & 0.01 & 0.01 &  & 0.61 \\ 
$\Delta p_{0}^{SS} \, \text{Run-I}$ & 0.04 & 0.01 & 0.07 & 0.00 & 0.00 & 0.00 & 0.10 &  & 0.13 \\ 
$\Delta p_{1}^{SS} \, \text{Run-I}$ & 0.03 & 0.04 & 0.04 & 0.00 & 0.01 & 0.00 & 0.12 &  & 0.14 \\ 
$\epsilon_{\text{tag}}^{SS} \, \text{Run-I}$ & 0.02 & 0.02 & 0.10 & 0.00 & 0.00 & 0.00 & 0.01 &  & 0.11 \\ 
$\Delta \epsilon_{\text{tag}}^{SS} \, \text{Run-I}$ & 0.04 & 0.03 & 0.02 & 0.00 & 0.04 & 0.03 & 0.01 &  & 0.07 \\ 
$p_{0}^{OS} \, \text{Run-II}$ & 0.02 & 0.20 & 0.01 & 0.02 & 0.97 & 0.04 & 0.00 &  & 0.99 \\ 
$p_{1}^{OS} \, \text{Run-II}$ & 0.02 & 0.08 & 0.10 & 0.01 & 0.60 & 0.04 & 0.00 &  & 0.61 \\ 
$\Delta p_{0}^{OS} \, \text{Run-II}$ & 0.07 & 0.08 & 0.13 & 0.00 & 0.22 & 0.01 & 0.00 &  & 0.27 \\ 
$\Delta p_{1}^{OS} \, \text{Run-II}$ & 0.02 & 0.03 & 0.16 & 0.00 & 0.08 & 0.01 & 0.00 &  & 0.18 \\ 
$\epsilon_{\text{tag}}^{OS} \, \text{Run-II}$ & 0.01 & 0.16 & 0.04 & 0.00 & 0.11 & 0.00 & 0.00 &  & 0.20 \\ 
$\Delta \epsilon_{\text{tag}}^{OS} \, \text{Run-II}$ & 0.05 & 0.05 & 0.10 & 0.00 & 0.03 & 0.03 & 0.00 &  & 0.13 \\ 
$p_{0}^{SS} \, \text{Run-II}$ & 0.10 & 0.06 & 0.05 & 0.01 & 0.61 & 0.02 & 0.00 &  & 0.62 \\ 
$p_{1}^{SS} \, \text{Run-II}$ & 0.01 & 0.07 & 0.06 & 0.02 & 0.63 & 0.04 & 0.00 &  & 0.64 \\ 
$\Delta p_{0}^{SS} \, \text{Run-II}$ & 0.07 & 0.02 & 0.01 & 0.00 & 0.04 & 0.01 & 0.00 &  & 0.09 \\ 
$\Delta p_{1}^{SS} \, \text{Run-II}$ & 0.11 & 0.05 & 0.14 & 0.00 & 0.04 & 0.01 & 0.00 &  & 0.19 \\ 
$\epsilon_{\text{tag}}^{SS} \, \text{Run-II}$ & 0.03 & 0.03 & 0.09 & 0.00 & 1.68 & 0.00 & 0.00 &  & 1.68 \\ 
$\Delta \epsilon_{\text{tag}}^{SS} \, \text{Run-II}$ & 0.01 & 0.03 & 0.01 & 0.00 & 0.21 & 0.03 & 0.00 &  & 0.21 \\ 
$A_{P} \, \text{Run-II}$ & 0.04 & 0.02 & 0.17 & 0.00 & 0.19 & 0.03 & 0.01 &  & 0.26 \\ 
$\Delta m_{s}$ & 0.01 & 0.11 & 0.02 & 0.00 & 0.13 & 0.69 & 0.02 & 0.67 & 0.98 \\ 
\hline
\hline
\end{tabular}

}
\label{tab:sigSys}
\end{table}


\begin{sidewaystable}[h]
\centering
\caption{Systematic uncertainties on the fit parameters of the full time-dependent amplitude fit to $B_s \to D_s K\pi \pi$ data in units of statistical standard deviations.}
\resizebox{\linewidth}{!}{
        \renewcommand{\arraystretch}{1.5}
        \begin{tabular}{l  c  c  c  c  c  c  c  c  | c }
\hline
\hline
Fit Parameter & Fit-bias & Background & Correlations & Acceptance & Resolution & Decay-time bias & Asymmetries & Mom./z-Scale &  Total  \\ 
\hline
$p_{0}^{OS} \, \text{Run-I}$ & 0.05 & 0.09 & 0.03 & 0.01 & 1.00 & 0.02 & 0.00 &  & 1.00 \\ 
$p_{1}^{OS} \, \text{Run-I}$ & 0.01 & 0.13 & 0.06 & 0.01 & 1.04 & 0.02 & 0.01 &  & 1.05 \\ 
$\Delta p_{0}^{OS} \, \text{Run-I}$ & 0.14 & 0.03 & 0.03 & 0.00 & 0.02 & 0.01 & 0.15 &  & 0.21 \\ 
$\Delta p_{1}^{OS} \, \text{Run-I}$ & 0.07 & 0.06 & 0.16 & 0.00 & 0.03 & 0.01 & 0.15 &  & 0.24 \\ 
$\epsilon_{\text{tag}}^{OS} \, \text{Run-I}$ & 0.06 & 0.17 & 0.12 & 0.00 & 0.00 & 0.00 & 0.01 &  & 0.22 \\ 
$\Delta \epsilon_{\text{tag}}^{OS} \, \text{Run-I}$ & 0.04 & 0.01 & 0.05 & 0.00 & 0.06 & 0.02 & 0.01 &  & 0.09 \\ 
$p_{0}^{SS} \, \text{Run-I}$ & 0.03 & 0.03 & 0.15 & 0.01 & 0.56 & 0.02 & 0.00 &  & 0.58 \\ 
$p_{1}^{SS} \, \text{Run-I}$ & 0.10 & 0.03 & 0.04 & 0.01 & 0.60 & 0.01 & 0.01 &  & 0.61 \\ 
$\Delta p_{0}^{SS} \, \text{Run-I}$ & 0.04 & 0.01 & 0.07 & 0.00 & 0.00 & 0.00 & 0.10 &  & 0.13 \\ 
$\Delta p_{1}^{SS} \, \text{Run-I}$ & 0.03 & 0.04 & 0.04 & 0.00 & 0.01 & 0.00 & 0.12 &  & 0.14 \\ 
$\epsilon_{\text{tag}}^{SS} \, \text{Run-I}$ & 0.02 & 0.02 & 0.10 & 0.00 & 0.00 & 0.00 & 0.01 &  & 0.11 \\ 
$\Delta \epsilon_{\text{tag}}^{SS} \, \text{Run-I}$ & 0.04 & 0.03 & 0.02 & 0.00 & 0.04 & 0.03 & 0.01 &  & 0.07 \\ 
$p_{0}^{OS} \, \text{Run-II}$ & 0.02 & 0.20 & 0.01 & 0.02 & 0.97 & 0.04 & 0.00 &  & 0.99 \\ 
$p_{1}^{OS} \, \text{Run-II}$ & 0.02 & 0.08 & 0.10 & 0.01 & 0.60 & 0.04 & 0.00 &  & 0.61 \\ 
$\Delta p_{0}^{OS} \, \text{Run-II}$ & 0.07 & 0.08 & 0.13 & 0.00 & 0.22 & 0.01 & 0.00 &  & 0.27 \\ 
$\Delta p_{1}^{OS} \, \text{Run-II}$ & 0.02 & 0.03 & 0.16 & 0.00 & 0.08 & 0.01 & 0.00 &  & 0.18 \\ 
$\epsilon_{\text{tag}}^{OS} \, \text{Run-II}$ & 0.01 & 0.16 & 0.04 & 0.00 & 0.11 & 0.00 & 0.00 &  & 0.20 \\ 
$\Delta \epsilon_{\text{tag}}^{OS} \, \text{Run-II}$ & 0.05 & 0.05 & 0.10 & 0.00 & 0.03 & 0.03 & 0.00 &  & 0.13 \\ 
$p_{0}^{SS} \, \text{Run-II}$ & 0.10 & 0.06 & 0.05 & 0.01 & 0.61 & 0.02 & 0.00 &  & 0.62 \\ 
$p_{1}^{SS} \, \text{Run-II}$ & 0.01 & 0.07 & 0.06 & 0.02 & 0.63 & 0.04 & 0.00 &  & 0.64 \\ 
$\Delta p_{0}^{SS} \, \text{Run-II}$ & 0.07 & 0.02 & 0.01 & 0.00 & 0.04 & 0.01 & 0.00 &  & 0.09 \\ 
$\Delta p_{1}^{SS} \, \text{Run-II}$ & 0.11 & 0.05 & 0.14 & 0.00 & 0.04 & 0.01 & 0.00 &  & 0.19 \\ 
$\epsilon_{\text{tag}}^{SS} \, \text{Run-II}$ & 0.03 & 0.03 & 0.09 & 0.00 & 1.68 & 0.00 & 0.00 &  & 1.68 \\ 
$\Delta \epsilon_{\text{tag}}^{SS} \, \text{Run-II}$ & 0.01 & 0.03 & 0.01 & 0.00 & 0.21 & 0.03 & 0.00 &  & 0.21 \\ 
$A_{P} \, \text{Run-II}$ & 0.04 & 0.02 & 0.17 & 0.00 & 0.19 & 0.03 & 0.01 &  & 0.26 \\ 
$\Delta m_{s}$ & 0.01 & 0.11 & 0.02 & 0.00 & 0.13 & 0.69 & 0.02 & 0.67 & 0.98 \\ 
\hline
\hline
\end{tabular}

}
\label{tab:sigSys2}
\end{sidewaystable}




