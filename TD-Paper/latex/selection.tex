\section{Selection of signal candidates}
\label{sec:Selection}

In the first selection step, charged kaons and pions are reconstructed to form a $\Ds$ candidate in the decay modes $\Ds\to\kaon\kaon\pion$, $\Ds\to\kaon\pion\pion$ and $\Ds\to\pion\pion\pion$.
These candidates are subsequently combined with a kaon and two pions or three pions from the secondary vertex to form $\Bs\to\Ds\kaon\pion\pion$ or $\Bs\to\Ds\pion\pion\pion$ candidates.
The resolution of the invariant mass of $\Bs$ candidates, as well as the decay-time resolution, are improved using a kinematic fit~\cite{DTF} 
where the $\Bs$ candidate is constrained to the PV for which it has the smallest IP signficance and the mass of the $\Ds$ is constrained to the world average. \newline
Further kinematic vetoes and requirements on the particle identification (PID) information are used to distinguish the different $\Ds$ final states and isolate them from physical background, such as decays of $\PLambda_{c}$ or $\Dpm$.
The most abundant final state of the $\Ds$ meson, $\Ds\to\kaon\kaon\pion$, is subdivided into $\Ds\to\phi\pi$, $\Ds\to\Kstarz\kaon$ and $\Ds\to(\kaon\kaon\pion)_{\text{non-res}}$, 
where the narrow $\phi$ and $\Kstarz$ resonances allow for looser requirements on the PID variables for those candidates. 
Additional kinematic selections are applied to the other three hadrons to suppress background from physical cross-feed, e.g. the $\Bs\to\Ds\Ds$ decay.
The majority of criteria used in this analysis are guided by the selection procedures implemented in~\cite{LHCb-PAPER-2012-033,Aaij:2017lff}. \newline 
To discriminate signal candidates and combinatorial background, a boosted decision tree~(BDT)~\cite{Breiman,AdaBoost} implemented in the TMVA toolkit~\cite{Hocker:2007ht,*TMVA4} to separate signal from background is used.
The decision tree is trained using background-subtacted $\Bs\to\Ds\pion\pion\pion$ data as signal proxy, while the upper mass sideband of $\Bs\to\Ds\kaon\pion\pion$ candidates ($m_{\Bs} > 5500 \mevcc$) is used as background proxy.
Kinematic quantities of the $\Bs$, the $\Ds$ and the reconstructed kaons and pions are used as discriminating variables for the training, while no information from the PID system is taken.
The working point of the decision tree is chosen to optimize the significance of the $\Bs\to\Ds\kaon\pion\pion$ signal.
After the full selection procedure is applied, approximately 1.5 $\%$ of events contain more than one signal candidate, off which all are used for the analysis.


