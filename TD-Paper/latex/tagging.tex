\section{Flavour tagging}
\label{sec:Tagging}

To identify the initial flavour state of the $\Bs$ meson,
a number of flavour tagging algorithms are used that either exploit the pair-wise production of \bquark quarks and determine the flavour of the non-signal b-hadron produced in the event (opposite site, OS)
or use particles produced in the fragmentation of the signal candidate $\Bs$/$\Bsb$ (same side, SS).
The same side kaon tagger searches for the charge of an additional kaon that accompanies the fragmentation of the signal $\Bs$ or $\Bsb$ candidate~\cite{Aaij:2016psi}.
For the opposite site tagger~\cite{LHCb-PAPER-2011-027}, five different tagging algorithms are chosen:
the algorithms that use the charge of an electron or a muon from semi-leptonic B decays,
the tagger that uses the charge of a kaon from a b $\to$ c $\to$ s decay chain,
the algorithm which reconstructs opposite-side charm hadrons from a number of \cquark-decays
and the algorithm that determines the $\Bs$/$\Bsb$ candidate flavour from the charge of a secondary vertex, reconstructed from the OS \bquark-decay product.
All five taggers are then combined into a single OS tagger. \newline
Every tagging algorithm is prone to misidentify the signal candidate at a certain mistag rate $\omega  = (\text{wrong tags})/ (\text{all tags})$. 
This might be caused by particle misidentification, flavour oscillation of the neutral opposite site B-meson or by the selection of tracks from the underlying event.
An imperfect determination of the $\Bs$ production flavour dilutes the observed \CP asymmetry by $D_{tag} = 1 - 2\omega$.
Therefore, the statistical precision with which the \CP asymmetry can be measured scales as the inverse square root of the effective tagging efficiency:
\begin{equation}
\label{eq: taggingEfficiency}
\epsilon_{eff} = \epsilon_{tag}(1 - 2\omega)^{2},
\end{equation}
where $\epsilon_{tag}$ is the fraction of tagged signal candidates.\newline
For each $\Bs$/$\Bsb$ candidate, the tagging algorithms provide a prediction for the mistag probability $\eta$ based on the output of neural networks that take various variables, 
such as the kinematics of the tagging particles, as input.
These are trained on either simulated or \textsf{sWeighted} samples of flavour specific control channels ($\Bs\to\Dsm\pip$ (SS algorithm) and $\Bu\to\jpsi\Kp$ (OS algorithms))
and are optimized for highest $\epsilon_{eff}$ on data.
Utilizing flavour-specific final states, the estimated mistag $\eta$ of each tagger has to be calibrated to match the actual mistag probability $\omega$.
For the calibration, a linear model
\begin{equation}
\label{eq: mistagCalibration}
\omega(\eta) = p_{0} + p_{1} \cdot (\eta - \langle \eta \rangle), 
\end{equation}
is used where $\langle \eta \rangle$ is the average estimated mistag probability.
A perfectly calibrated tagger would lead to $\omega(\eta) = \eta$ and one would expect $p_{1} = 1$ and $p_{0} = \langle \eta \rangle$.
Due to the different interaction cross-sections of oppositely charged kaons, the tagging calibration parameters depend on the initial state flavour of the $\Bs$.
Therefore, the flavour asymmetry parameters $\Delta p_{0}$, $\Delta p_{1}$ and $\Delta\epsilon_{tag}$ are introduced and defined as the difference of the corresponding values for $\Bs$ and $\Bsb$ mesons.

\begin{table}[h]
\centering
%\scriptsize
\caption{The flavour tagging performances for the used OS taggers for Run-I data.}
\resizebox{\linewidth}{!}{
        \begin{table}
\centering
\begin{tabular}{rlllll}
\multicolumn{1}{c}{Tagger} & \multicolumn{1}{c}{$\epsilon$} & \multicolumn{1}{c}{$\omega$} & \multicolumn{1}{c}{$\epsilon \langle D^2 \rangle = \epsilon \left( 1 - 2 \omega \right)^2$} \\ 
OS $\mu$& $(8.775\pm0.207)\%$& $(28.935\pm0.180(\textrm{stat})\pm2.288(\textrm{cal}))\%$& $(1.558\pm0.045(\textrm{stat})\pm0.338(\textrm{cal}))\%$\\
OS $e$& $(3.191\pm0.129)\%$& $(28.778\pm0.366(\textrm{stat})\pm3.636(\textrm{cal}))\%$& $(0.575\pm0.031(\textrm{stat})\pm0.197(\textrm{cal}))\%$\\
OS $K$ NN& $(32.099\pm0.342)\%$& $(38.405\pm0.094(\textrm{stat})\pm1.152(\textrm{cal}))\%$& $(1.726\pm0.033(\textrm{stat})\pm0.343(\textrm{cal}))\%$\\
Vertex Charge& $(21.797\pm0.302)\%$& $(35.672\pm0.092(\textrm{stat})\pm1.480(\textrm{cal}))\%$& $(1.790\pm0.034(\textrm{stat})\pm0.370(\textrm{cal}))\%$\\
\end{tabular}
\end{table}

}
\label{tab:OS_Run1}
%\end{table}
%
%\begin{table}[h]
\caption{The flavour tagging performances for the used OS taggers for Run-II data.}
\resizebox{\linewidth}{!}{
        \begin{table}
\centering
\begin{tabular}{rlllll}
\multicolumn{1}{c}{Tagger} & \multicolumn{1}{c}{$\epsilon$} & \multicolumn{1}{c}{$\omega$} & \multicolumn{1}{c}{$\epsilon \langle D^2 \rangle = \epsilon \left( 1 - 2 \omega \right)^2$} \\ 
OS $\mu$& $(8.775\pm0.207)\%$& $(28.935\pm0.180(\textrm{stat})\pm2.288(\textrm{cal}))\%$& $(1.558\pm0.045(\textrm{stat})\pm0.338(\textrm{cal}))\%$\\
OS $e$& $(3.191\pm0.129)\%$& $(28.778\pm0.366(\textrm{stat})\pm3.636(\textrm{cal}))\%$& $(0.575\pm0.031(\textrm{stat})\pm0.197(\textrm{cal}))\%$\\
OS $K$ NN& $(32.099\pm0.342)\%$& $(38.405\pm0.094(\textrm{stat})\pm1.152(\textrm{cal}))\%$& $(1.726\pm0.033(\textrm{stat})\pm0.343(\textrm{cal}))\%$\\
Vertex Charge& $(21.797\pm0.302)\%$& $(35.672\pm0.092(\textrm{stat})\pm1.480(\textrm{cal}))\%$& $(1.790\pm0.034(\textrm{stat})\pm0.370(\textrm{cal}))\%$\\
\end{tabular}
\end{table}

}
\label{tab:OS_Run2}
\end{table}

The OS electron, muon, kaon, charm and the secondary vertex charge tagging algorithms are individually calibrated and then combined into a singleOS tagger. 
We choose the flavour specific decay  $B_s \to D_s \pi\pi\pi$ as calibration mode due to the portability to the signal mode since its similarity wit the $B_s \to D_s K\pi\pi$ decay.
The calibration is performed separately for Run-I and Run-II data, while the OS-c tagger is not included for Run-I data since the statistics is too low.
Tables \ref{tab:OS_Run1} and \ref{tab:OS_Run2} list the measured tagging performances.\newline
The combined OS and SS-Kaon taggers are calibrated simultaneously by means of a fit to the decay-time distribution of background-subtracted $B_s \to D_s \pi\pi\pi$ candidates, as discussed in Sec.~\ref{sec:timeFit}.
In this fit, the predicted mistag probabilities $\eta_{OS}$ and $\eta_{SS}$ are included as per-event observables, effectively giving a larger weight to the events that have a lower mistag probability.
The tagger responses are combined into a single response on an event-by-event basis during the fit.
Tables \ref{tab:tagPerfRun1} and \ref{tab:tagPerfRun2} report the tagging performances for the OS and SS combination
considering three mutually exclusive categories of tagged events: OS only, SS only and both OS and SS.
The tagging calibration parameters are listed in Table~\ref{tab:normFitResults}.


\begin{table}[h]
\centering
\caption{The flavour tagging performances for only OS tagged, only SS tagged and both OS and SS tagged events for Run-I data.}
\begin{tabular}{c c c c}
\hline
\hline
$ B_s \to D_s \pi \pi \pi$ & $\epsilon_{tag} [\%]$ & $\langle \omega \rangle [\%] $ & $\epsilon_{eff} [\%]$ \\
\hline
Only OS & 14.74 $\pm$ 0.11 & 39.09 $\pm$ 0.80 & 1.25 $\pm$ 0.16\\
Only SS & 35.38 $\pm$ 0.18 & 44.26 $\pm$ 0.62 & 1.05 $\pm$ 0.18\\
Both OS-SS & 33.04 $\pm$ 0.30 & 37.33 $\pm$ 0.73 & 3.41 $\pm$ 0.33\\
\hline
Combined & 83.16 $\pm$ 0.37 & 40.59 $\pm$ 0.70 & 5.71 $\pm$ 0.40\\
\hline
\hline
\end{tabular}

\label{tab:tagPerfRun1}

\caption{The flavour tagging performances for only OS tagged, only SS tagged and both OS and SS tagged events for Run-II data.}
\begin{tabular}{c c c c}
\hline
\hline
$ B_s \to D_s K \pi \pi$ & $\epsilon_{tag} [\%]$ & $\langle \omega \rangle [\%] $ & $\epsilon_{eff} [\%]$ \\
\hline
Only OS & 12.99 $\pm$ 0.06 & 37.06 $\pm$ 0.51 & 1.28 $\pm$ 0.08\\
Only SS & 39.89 $\pm$ 0.10 & 42.92 $\pm$ 0.35 & 1.63 $\pm$ 0.11\\
Both OS-SS & 28.64 $\pm$ 0.14 & 35.50 $\pm$ 0.40 & 3.59 $\pm$ 0.16\\
\hline
Combined & 81.53 $\pm$ 0.18 & 39.38 $\pm$ 0.40 & 6.50 $\pm$ 0.21\\
\hline
\hline
\end{tabular}

\label{tab:tagPerfRun2}
\end{table}

